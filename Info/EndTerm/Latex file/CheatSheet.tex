%!TEX root = CheatSheet.tex

% --- PAQUETS DE BASE ---
\documentclass[10pt,a4paper]{article}
\usepackage[utf8]{inputenc} % Encodage des caractères
\usepackage[T1]{fontenc}      % Encodage des polices
\usepackage[french]{babel}    % Règles typographiques françaises
\usepackage[left=1cm, right=1cm, top=1cm, bottom=2.5cm]{geometry} % Marges de la page

% --- MISE EN PAGE & OUTILS ---
\usepackage{multicol}     % Texte sur plusieurs colonnes
\usepackage{xcolor}       % Gestion des couleurs
\usepackage{graphicx}     % Pour inclure des images
\usepackage{enumitem}     % Personnalisation des listes
\usepackage{titlesec}     % Personnalisation des titres de section
\usepackage{array}        % Amélioration des tableaux
\usepackage{tikz}         % Pour dessiner des diagrammes
\usepackage{float}        % Pour un placement précis des flottants (ex: tables, figures)
\usetikzlibrary{arrows}   % Pour les flèches dans les diagrammes
\usepackage{lstautogobble} % Pour la gestion automatique de l'indentation dans les listings

% --- POUR LE CODE ---
\usepackage{listings}     % Pour insérer du code source

% --- CONFIGURATION DU CODE JAVA (listings) ---
\definecolor{javared}{rgb}{0.6,0,0} % for strings
\definecolor{javagreen}{rgb}{0.25,0.5,0.35} % comments
\definecolor{javapurple}{rgb}{0.5,0,0.35} % keywords
\definecolor{javadocblue}{rgb}{0.25,0.35,0.75} % javadoc

\lstset{
  language=Java,
  basicstyle=\ttfamily\normalsize,
  keywordstyle=\color{javapurple}\bfseries,
  stringstyle=\color{javared},
  commentstyle=\color{javagreen},
  morecomment=[s][\color{javadocblue}]{/**}{*/},
  literate={é}{{\'{e}}}1 {à}{{\`{a}}}1 {è}{{\`{e}}}1 {ç}{{\c{c}}}1 {ê}{{\^{e}}}1 {ë}{{\¨{e}}}1 {ï}{{\¨{i}}}1 {ô}{{\^{o}}}1 {û}{{\^{u}}}1 {ù}{{\`{u}}}1,
  autogobble=true,
  numbers=none,
  tabsize=2,
  breaklines=true,
  frame=single,
  showstringspaces=false % Ne pas montrer les espaces dans les chaînes
}

% --- COMMANDES PERSONNELLES (Optionnel, pour plus tard) ---
% Exemple : \newcommand{\motcle}[1]{\textbf{#1}}
 % Importe toutes les configurations et paquets

\begin{document}

% ===================================================================
% SECTION 1: BASES ET PIÈGES
% ===================================================================
\section{Bases et Pièges Classiques}

\subsection{Comparaison}
\begin{itemize}
    \item \textbf{Primitifs} (\lstinline!int, double, boolean!) : Utiliser \lstinline!==!.
    \item \textbf{Objets} (\lstinline!String, Tableaux, Instances!) : 
    \begin{itemize}
        \item \lstinline!a == b! : Compare les \textbf{références} (adresses mémoire).
        \item \lstinline!a.equals(b)! : Compare le \textbf{contenu} (si redéfini).
    \end{itemize}
    \item \textbf{String Pool :} \lstinline!"A" == "A"! (true), mais \lstinline!new String("A") == "A"! (false). Toujours utiliser \lstinline!.equals()! pour les Strings !
\end{itemize}

\subsection{Tableaux vs ArrayList}
\begin{table}[!h]
    \centering
    \begin{tabular}{|>{\raggedright\arraybackslash}p{3.5cm}|>{\raggedright\arraybackslash}p{3.5cm}|}
    \hline
    \textbf{Tableau (Fixe)} & \textbf{ArrayList (Dynamique)} \\
    \hline
    \lstinline!int[] t = new int[5];! & \lstinline!ArrayList<Integer> l = ... ;! \\
    \hline
    \lstinline!t.length! (attribut) & \lstinline!l.size()! (méthode) \\
    \hline
    \lstinline!t[i]! & \lstinline!l.get(i)! \\
    \hline
    \lstinline!t[i] = val! & \lstinline!l.set(i, val)! \\
    \hline
    Immuable en taille & \lstinline!l.add(val)!, \lstinline!l.remove(i)! \\
    \hline
    \end{tabular}
    \caption{Tableau de comparaison entre les liste de taille fixe et dynamique}   
\end{table}



% ===================================================================
% SECTION 2: POO
% ===================================================================
\section{Programmation Orientée Objet}

\subsection{Constructeurs}
\begin{itemize}
    \item \textbf{Par défaut :} Fourni par Java \textit{si aucun autre constructeur n'existe}. Initialise à \lstinline!0/null/false!.
    \item \textbf{Chainage (this)} : \lstinline!this(...)! appelle un autre constructeur de la \textbf{même classe}. Doit être la \textbf{1ère ligne}.
    \item \textbf{Chainage (super)} : \lstinline!super(...)! appelle le constructeur de la \textbf{super-classe}. C'est la manifestation du \textbf{polymorphisme en construction} : l'objet se construit "de la base vers le dérivé".
    \item \textbf{Piège de la résolution dynamique} : Si un constructeur de la super-classe appelle une méthode qui est redéfinie dans la sous-classe, c'est la version de la \textbf{sous-classe} qui est exécutée, \textit{même si le constructeur de la sous-classe n'a pas encore été appelé}. Attention aux \lstinline!NullPointerException! si la méthode redéfinie utilise des champs initialisés uniquement dans le constructeur fils.
    \item \textbf{Règle d'or} : Si la 1ère ligne n'est pas \lstinline!this(...)! ou \lstinline!super(...)!, Java ajoute implicitement \lstinline!super()! (vide). Si le parent n'a pas de constructeur par défaut $\rightarrow$ Erreur de compilation.
    \item \textbf{Récursivité} : Un constructeur \textbf{ne peut pas} s'appeler lui-même (ni directement, ni indirectement). Le compilateur l'interdit pour éviter une boucle d'initialisation infinie. Il peut cependant appeler une méthode externe qui est, elle, récursive.
\end{itemize}

\subsection{Encapsulation \& Visibilité}
\begin{table}[h!]
    \centering
    \begin{tabular}{|l|c|c|c|c|}
    \hline
    Modifier & Classe & Pkg & Sous-Cl & Monde \\
    \hline
    \lstinline!public! & Oui & Oui & Oui & Oui \\
    \hline
    \lstinline!protected! & Oui & Oui & Oui & Non \\
    \hline
    \lstinline!(défaut)! & Oui & Oui & Non & Non \\
    \hline
    \lstinline!private! & Oui & Non & Non & Non \\
    \hline
    \end{tabular}
    \caption{Tableau de visibilité des modificateurs}
\end{table}


Lors de la redéfinition d'une méthode héritée, il est \textbf{interdit de réduire la visibilité} de la méthode par rapport à sa version dans la super-classe. La visibilité peut être maintenue ou étendue (ex: de \lstinline!protected! à \lstinline!public!), mais jamais restreinte (ex: de \lstinline!public! à \lstinline!protected! ou \lstinline!private!).

\subsection{Static vs Final}
\begin{itemize}
    \item \lstinline!static! : Appartient à la classe, pas à l'instance. Partagé par tous les objets.
    \begin{itemize}        
        \item \textbf{Variable static} : Partagée par toutes les instances de la classe. Accès via \lstinline!Classe.variable!.
        \item \textbf{Méthode static} : Appel via \lstinline!Classe.methode()!. N'a pas accès à \lstinline!this! ni à \lstinline!super! (pas d'instance).
        \item \textbf{Attention Sacha } : Une methde statique comme \lstinline|main| n'a pas accès aux attributs non statiques.
    \end{itemize}
    \item \lstinline!final! (variable) : Constante, assignée 1 seule fois (référence).
    \begin{itemize}
        \item \textbf{Primitif} : Valeur fixe.
        \item \textbf{Objet} : Référence fixe, mais contenu modifiable (sauf pour les objet immuable (\lstinline!String!)).
    \end{itemize}
\item \lstinline!final! (méthode) : Ne peut pas être redéfinie (\lstinline!@Override! interdit).
    \item \lstinline!final! (classe) : Ne peut pas avoir de sous-classes.
\end{itemize}

\subsection{Classes Imbriquées (Nested Classes)}
Une classe définie à l'intérieur d'une autre classe. Il en existe 4 types.

\subsubsection{Classe Interne (Inner Class)}
\begin{itemize}
    \item Définie sans le mot-clé \lstinline!static!.
    \item Liée à une instance de la classe externe. Ne peut exister sans elle.
    \item A accès à \textbf{tous} les membres de la classe externe, y compris \lstinline!private!.
    \item Syntaxe pour l'instancier depuis l'extérieur : \lstinline!Outer.Inner inner = outerInstance.new Inner();!
\end{itemize}
\begin{figure}[h!]
    \begin{lstlisting}
        class Outer {
            private int x = 10;
            class Inner {
                void display() {
                    System.out.println("x = " + x); // Accès direct
                }
            }
        }
    \end{lstlisting}
    \caption{Illustration d'une classe interne}
\end{figure}


\subsubsection{Classe Statique Imbriquée (Static Nested Class)}
\begin{itemize}
    \item Déclarée avec le mot-clé \lstinline!static!.
    \item Pas liée à une instance de la classe externe. C'est une classe de haut niveau "rangée" dans une autre.
    \item A accès uniquement aux membres \textbf{statiques} de la classe externe.
    \item Syntaxe pour l'instancier : \lstinline!Outer.Nested nested = new Outer.Nested();!
\end{itemize}
\begin{figure}[h!]
    \begin{lstlisting}[language=Java]
    class Outer {
        private static int y = 20;
        static class Nested {
            void display() {
                System.out.println("y = " + y); // Accès au static
            }
        }
    }
    \end{lstlisting}
    \caption{Illustration d'une classe imbriquée statique}
\end{figure}


\newpage
\subsection{Type énuméré (Enum)}
Un type énuméré est une classe spéciale qui représente un groupe de \textbf{constantes}. C'est la manière la plus propre et sûre de modéliser un ensemble fixe de valeurs (ex: jours de la semaine, états, couleurs).
\begin{itemize}
    \item \textbf{Type-Safe} : Le compilateur garantit qu'une variable de type \texttt{Jour} ne peut contenir que les valeurs définies (LUNDI, MARDI, etc.), évitant les erreurs liées à l'utilisation de \lstinline!String! ou \lstinline!int!.
    \item \textbf{Plus qu'un simple nom} : Une énumération est une vraie classe. Elle peut avoir des \textbf{attributs}, des \textbf{méthodes} et un \textbf{constructeur}.
    \item \textbf{Constructeur privé} : Le constructeur d'un enum est implicitement \textbf{privé}. On ne peut pas créer de nouvelles instances avec \lstinline!new!.
    \item \textbf{Utilisation} : On y accède via \lstinline!NomEnum.VALEUR! (ex: \lstinline!Feux.ROUGE!). Idéal pour les instructions \lstinline!switch!.
\end{itemize}
Voici un exemple d'énumération avec des attributs et des méthodes :
\begin{figure}[h!]
    \begin{lstlisting}
        public enum Feu {
            VERT("Vert", false),
            ORANGE("Orange", false),
            ROUGE("Rouge", true);

            private final String nom;
            private final boolean arretRequis;
            
            // Constructeur est implicitement private
            Feux(String nom, boolean arretRequis) {
                this.nom = nom;
                this.arretRequis = arretRequis;
            }

            public String getNom() {
                return nom;
            }

            public boolean doitArreter() {
                return arretRequis;
            }
        }

        // Utilisation
        Feux monFeu = Feux.VERT;
        if (monFeu.doitArreter()) {
            // ...
        }
    \end{lstlisting}
    \caption{Illustration de le déclaration et de l'utilisation d'un type énuméré}
\end{figure}

% ===================================================================
% SECTION 3: HÉRITAGE & POLYMORPHISME
% ===================================================================
\newpage
\section{Héritage et Polymorphisme}

\subsection{Définitions}
\begin{itemize}
    \item \textbf{Surcharge (Overload)} : Même nom, \textbf{paramètres différents}. Résolu à la \textbf{compilation} (type statique).
    \item \textbf{Redéfinition (Override)} : Même signature, comportement différent dans sous-classe. Résolu à l'\textbf{exécution} (type dynamique).
\end{itemize}

\subsection{Résolution des liens}
Soit \lstinline!A a = new B();! (B étend A).
\begin{enumerate}
    \item \textbf{Type Statique} (A) : Détermine ce qu'on a le \textit{droit} d'appeler. Si la méthode n'existe pas dans A $\rightarrow$ Erreur compil.
    \item \textbf{Type Dynamique} (B) : Détermine quelle version est exécutée. Si B redéfinit la méthode, c'est celle de B qui tourne.
    \item \textbf{Attributs} : Toujours liés au type \textbf{statique} (Pas de polymorphisme sur les attributs !).
    \item Les méthodes \lstinline!static! sont des liens résolus statiquement.
\end{enumerate}

\begin{figure}[h!]
    \begin{lstlisting}
        class Animal {
            public void manger() {
                System.out.println("L'animal mange.");
            }
        }

        class Chien extends Animal {
            @Override
            public void manger() {
                System.out.println("Le chien mange sa gamelle.");
            }

            public void aboyer() {
                System.out.println("Woof !");
            }
        }

        public class TestPolymorphisme {
            public static void main(String[] args) {
                // Type statique: Animal, Type dynamique: Chien
                Animal monAnimal = new Chien();

                // OK: manger() existe dans Animal.
                // La version de Chien est appelée.
                monAnimal.manger();

                // ERREUR DE COMPILATION: aboyer() n'existe pas dans Animal.
                // monAnimal.aboyer();

                // Pour y accéder, il faut une référence de type Chien.
                Chien monChien = (Chien) monAnimal;
                monChien.aboyer(); // OK après cast.
            }
        }
    \end{lstlisting}
    \caption{Exemple dévelopé pour le point 1 de la résolution des liens}
\end{figure}


\subsection{Classes Abstraites vs Interfaces}
\begin{table}[H]
    \centering
    \begin{tabular}{|>{\raggedright\arraybackslash}p{3.5cm}|>{\raggedright\arraybackslash}p{3.5cm}|}
    \hline
    \textbf{Classe Abstraite} & \textbf{Interface} \\
    \hline
    \lstinline!extends! (1 seule) & \lstinline!implements! (N) \\
    \hline
    Peut avoir état (attributs) & Pas d'attributs (sauf constantes static final) \\
    \hline
    Constructeurs possibles & Pas de constructeurs \\
    \hline
    Méthodes privées/protected OK & Méthodes public par défaut \\
    \hline
    Relation "EST-UN" & Relation "SE COMPORTE COMME" \\
    \hline
    \end{tabular}
    \caption{Différences entre classe abstraite et interface.}
\end{table}

\textit{Java 8+} : Les interfaces peuvent avoir des méthodes \lstinline!default! (avec corps) et \lstinline!static!.
\newline\textbf{Attention aussi :} Une méthode \lstinline!default! dans une interface \textbf{ne peut pas} être déclarée \lstinline!final!. Une méthode \lstinline!default! est conçue pour être redéfinie, ce qui est incompatible avec le modificateur \lstinline!final!.
\newline\textbf{Attention :} Les variables dans une interface sont \textbf{toujours} \lstinline!public static final! (des constantes), on ne peut pas avoir de variable d'instance ou avec la visibilité \lstinline!default!.


% ===================================================================
% SECTION 4: EXCEPTIONS
% ===================================================================
\section{Exceptions}

\subsection{Hiérarchie des Exceptions et Règle Fondamentale}

La règle de base du compilateur est la suivante : toute classe qui hérite de \lstinline|Throwable| est considérée comme \textbf{checked}, \underline{sauf} :
\begin{itemize}
    \item La classe \lstinline|Error| et ses descendantes (erreurs système graves).
    \item La classe \lstinline|RuntimeException| et ses descendantes (erreurs de programmation).
\end{itemize}

C'est pourquoi \textbf{\lstinline|throw new Exception()|} est une opération \textit{checked} : la classe \lstinline|Exception| n'est ni une \lstinline|Error|, ni une \lstinline|RuntimeException|. Le schéma ci-dessous clarifie cette hiérarchie et la distinction.

\begin{figure}[H]
    \centering
    \begin{tikzpicture}[
        treenode/.style = {align=center, inner sep=3pt, text centered, font=\sffamily, rectangle, rounded corners, draw},
        level 1/.style={sibling distance=5.5cm, level distance=1.5cm},
        level 2/.style={sibling distance=4.5cm, level distance=1.5cm},
        level 3/.style={level distance=1.8cm},
        level 4/.style={}
        edge from parent/.style={draw, -latex}
    ]
    \node[treenode, fill=yellow!20] {Throwable}
        child { node[treenode, fill=gray!30] {\lstinline|Error|} }
        child { node[treenode, fill=blue!20] (E) {Exception \lstinline|Exception|}
            child { node[treenode, fill=blue!20] {\lstinline|IOException|, \lstinline|SQLException|, etc.\\ \textbf{(Checked)}} }
            child { node[treenode, fill=red!20] {\lstinline|RuntimeException| \\ \textbf{(Unchecked)}}
                child { node[treenode, fill=red!20, font=\sffamily\small] {\lstinline|NullPointerException| \\ \lstinline|IllegalArgumentException| \\ etc. \\ \textbf{(Unchecked)}} }
            }
        };
    \end{tikzpicture}
    \caption{Illustration de hiérarchie de classe}
\end{figure}



\begin{itemize}
    \item \textbf{Throwable} : Classe racine de tout ce qui peut être "lancé".
    \item \textbf{Error} : Erreurs graves (ex: \lstinline!OutOfMemoryError!), non destinées à être attrapées.
    \item \textbf{Exception} : Erreurs applicatives.
    \begin{itemize}
        \item \textbf{Branche `RuntimeException` (Unchecked)} : Erreurs de logique qui ne demandent pas de \lstinline!try-catch! obligatoire (ex: \lstinline!NullPointerException!). Le compilateur les ignore.
        \item \textbf{Les autres branches (Checked)} : Erreurs prévisibles que le compilateur \textbf{force} à gérer (ex: \lstinline!IOException!).
    \end{itemize}
\end{itemize}

\subsection{Rôle du Compilateur }
Le compilateur Java applique une règle stricte pour les \textbf{checked exceptions} : la règle du "Catch or Specify" (Attraper ou Déclarer).

Si un morceau de code peut lever une \textit{checked exception}, le compilateur \textbf{refusera de compiler} tant que vous n'aurez pas fait l'une des deux choses suivantes :
\begin{enumerate}
    \item \textbf{Attraper (Catch)} : Encadrer le code à risque dans un bloc \lstinline!try-catch!. C'est vous qui gérez le problème.
    \item \textbf{Déclarer (Specify)} : Ajouter \lstinline!throws NomDeLException! à la signature de la méthode. Vous déléguez ainsi la responsabilité de la gestion à la méthode appelante.
\end{enumerate}

\textbf{Question clé :} Le compilateur râle-t-il si on utilise \lstinline!throw! sans \lstinline!throws! ?
\begin{itemize}
    \item \textbf{OUI}, si l'exception est \textit{checked}. Le code ci-dessous ne compile pas car \lstinline!Exception! est checked et non déclarée.
    \begin{lstlisting}[language=Java]
// ERREUR DE COMPILATION !
void methode() {
    throw new Exception("Ceci est une checked exception");
}
    \end{lstlisting}
    Pour corriger : \lstinline!void methode() throws Exception { ... }!

    \item \textbf{NON}, si l'exception est \textit{unchecked} (\lstinline!RuntimeException! ou une de ses filles). Le code ci-dessous compile parfaitement.
    \begin{lstlisting}[language=Java]
// OK, COMPILE !
void methode() {
    throw new RuntimeException("Ceci est une unchecked exception");
}
    \end{lstlisting}
\end{itemize}

\subsection{Blocs de Gestion}
\begin{itemize}
    \item \textbf{try} : Contient le code qui peut lever une exception.
    \item \textbf{catch} : Attrape et gère une exception. On peut avoir plusieurs blocs \lstinline!catch! pour différents types d'exceptions. \textbf{Important :} Toujours attraper les exceptions les plus spécifiques avant les plus générales.
    \item \textbf{finally} : Bloc de code \textbf{exécuté dans tous les cas}, qu'une exception soit levée ou non, et même si un \lstinline!return! est présent dans le bloc \lstinline!try!. Très utilisé pour libérer des ressources (ex: fermer un fichier).
    \item \textbf{throw} : Pour lancer manuellement une exception (\lstinline!throw new MyException();!).
    \item \textbf{throws} : Dans la signature d'une méthode pour déclarer les \textit{checked exceptions} qu'elle peut lancer.
\end{itemize}
\begin{figure}[H]
    \begin{lstlisting}
    void readFile() throws IOException {
        FileReader reader = null;
        try {
            reader = new FileReader("file.txt");
            // ... code de lecture
        } catch (FileNotFoundException e) {
            // Gérer le cas où le fichier n'existe pas
            System.out.println("Fichier non trouvé !");
        } finally {
            if (reader != null) {
                reader.close(); // Libération des ressources
            }
        }
    }
    \end{lstlisting}
    \caption{Exemple de bloc de gestion des examens}
\end{figure}



% ===================================================================
% SECTION 5: MODÈLES DE CONCEPTION
% ===================================================================
\section{Modèles de Conception (Exam)}

\subsection{Checklist Exercice Conception}
Si on demande de "modéliser" (ex: Termites) :
\begin{enumerate}
    \item \textbf{Identifier les classes} : Noms communs (Termite, Brindille).
    \item \textbf{Héritage} : "Est un type de..." $\rightarrow$ \lstinline!extends!.
    \item \textbf{Interface} : "Peut être..." ou "Se comporte comme..." $\rightarrow$ \lstinline!implements!.
    \item \textbf{Encapsulation} : Attributs \lstinline!private!. Accès via Getters/Setters \textbf{si nécessaire}.
    \item \textbf{Collections} : "Contient plusieurs..." $\rightarrow$ \texttt{ArrayList<Type>}.
\end{enumerate}

\subsection{Copie Profonde}
Pour copier un objet qui contient des objets (références) :
\begin{lstlisting}
    public class MaClasse1{
        private MaClasse2 objet;
        private int primitif;

        public MaClasse(MaClasse1 autre) {
            this.primitif = autre.primitif;
            // COPIE PROFONDE:
            this.objet = new MaClasse2(autre.objet); //Il faut que MaClasse2 ait un constructeur de copie
            
            //Remarque : on peut faire autre.objet même si l'attribut est private car on est dans la même classe et Java autorise l'accès aux attributs privés entre instances de la même classe!
        }
    }

\end{lstlisting}
Ne jamais faire \lstinline|this.objet = autre.objet| (copie de surface, danger !).


% ===================================================================
% SECTION 6: ASTUCES DIVERSES
% ===================================================================
\section{Astuces du Corrigé 18-19}
\begin{itemize}
    \item \textbf{Passage de paramètres (Toujours par valeur)} : Le passage de paramètres à une méthode en Java se fait toujours ``par valeur''. Cependant, le comportement diffère entre les types primitifs et les objets.
    \begin{itemize}
        \item \textbf{Types primitifs} (\lstinline|int|, \lstinline|double|, etc.) : La \emph{valeur} de la variable est copiée. Toute modification de la copie à l'intérieur de la méthode n'affecte pas la variable originale.\\[\medskipamount]
        \begin{minipage}{\linewidth}
        \begin{lstlisting}[language=Java]
void method(int i) {
    i = 10; // Ne change pas la variable originale à l'extérieur
}
        \end{lstlisting}
        \end{minipage}

        \item \textbf{Objets} (\lstinline!String!, \texttt{A}, etc.) : La \emph{valeur de la référence} (l'adresse de l'objet) est copiée. Les deux références (originale et copie) pointent vers le \textbf{même objet}.
        \begin{itemize}
            \item Si on \textbf{modifie l'état} de l'objet via cette référence copiée (ex: \texttt{a.setA(5)}), la modification est visible à l'extérieur.
            \item Si on \textbf{réassigne la référence} dans la méthode (ex: \texttt{a = new A(10)}), seule la copie de la référence est changée pour pointer vers un nouvel objet. La variable originale à l'extérieur continue de pointer vers l'objet initial.
        \end{itemize}

        \begin{minipage}{\linewidth}
        \begin{lstlisting}[language=Java]
void m(A a_param, int i_param) {
    ++i_param; // Ne change pas l'original
    a_param.setA(a_param.getA() + 1); // CHANGE l'objet original
    a_param = new A(10); // NE CHANGE PAS la variable originale
}
        \end{lstlisting}
        \end{minipage}
    \end{itemize}
    \item \textbf{Math} : \texttt{1/2 = 0} (entiers), \texttt{1.0/2 = 0.5}.
    \item \textbf{Immutable} : String est immuable. \texttt{s.toUpperCase()} ne change pas s, il renvoie un nouveau String.
    \item \textbf{ArrayList} : \texttt{List<Truc> l = new ArrayList<>();} (Coder vers l'interface List, pas l'implémentation ArrayList).
\end{itemize}
\begin{itemize}
   \item Pour déclarer une variable : \lstinline!int i = 0;!
   \item Pour appeler une méthode : \lstinline!obj.methode(); i++!
\end{itemize}
\end{document}