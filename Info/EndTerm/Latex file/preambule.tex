%!TEX root = CheatSheet.tex

% --- PAQUETS DE BASE ---
\documentclass[10pt,a4paper]{article}
\usepackage[utf8]{inputenc} % Encodage des caractères
\usepackage[T1]{fontenc}      % Encodage des polices
\usepackage[french]{babel}    % Règles typographiques françaises
\usepackage[left=1cm, right=1cm, top=1cm, bottom=2.5cm]{geometry} % Marges de la page

% --- MISE EN PAGE & OUTILS ---
\usepackage{multicol}     % Texte sur plusieurs colonnes
\usepackage{xcolor}       % Gestion des couleurs
\usepackage{graphicx}     % Pour inclure des images
\usepackage{enumitem}     % Personnalisation des listes
\usepackage{titlesec}     % Personnalisation des titres de section
\usepackage{array}        % Amélioration des tableaux
\usepackage{tikz}         % Pour dessiner des diagrammes
\usepackage{float}        % Pour un placement précis des flottants (ex: tables, figures)
\usetikzlibrary{arrows}   % Pour les flèches dans les diagrammes
\usepackage{lstautogobble} % Pour la gestion automatique de l'indentation dans les listings

% --- POUR LE CODE ---
\usepackage{listings}     % Pour insérer du code source

% --- CONFIGURATION DU CODE JAVA (listings) ---
\definecolor{javared}{rgb}{0.6,0,0} % for strings
\definecolor{javagreen}{rgb}{0.25,0.5,0.35} % comments
\definecolor{javapurple}{rgb}{0.5,0,0.35} % keywords
\definecolor{javadocblue}{rgb}{0.25,0.35,0.75} % javadoc

\lstset{
  language=Java,
  basicstyle=\ttfamily\normalsize,
  keywordstyle=\color{javapurple}\bfseries,
  stringstyle=\color{javared},
  commentstyle=\color{javagreen},
  morecomment=[s][\color{javadocblue}]{/**}{*/},
  literate={é}{{\'{e}}}1 {à}{{\`{a}}}1 {è}{{\`{e}}}1 {ç}{{\c{c}}}1 {ê}{{\^{e}}}1 {ë}{{\¨{e}}}1 {ï}{{\¨{i}}}1 {ô}{{\^{o}}}1 {û}{{\^{u}}}1 {ù}{{\`{u}}}1,
  autogobble=true,
  numbers=none,
  tabsize=2,
  breaklines=true,
  frame=single,
  showstringspaces=false % Ne pas montrer les espaces dans les chaînes
}

% --- COMMANDES PERSONNELLES (Optionnel, pour plus tard) ---
% Exemple : \newcommand{\motcle}[1]{\textbf{#1}}
