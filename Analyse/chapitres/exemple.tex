% !TEX root = ../Analyse1.tex
% (Cette ligne aide VSCode à savoir quel fichier maître compiler)

\chapter{Chapitre de Test}

\section{Première Section}

\subsection{Test des Encadrés}

% --- Test de la Définition ---
% Notez les accolades {} autour du titre, à cause des virgules
\begin{definition}[{Test de titre, avec virgules}]
Ceci est une définition (encadré rouge).
La numérotation devrait être 1.1.1.
Prenons $a$ t.q. $a \in \mathbb{R}$.
\end{definition}

% --- Test du Théorème ---
\begin{theorem}[{Test de Théorème}]
Ceci est un théorème (encadré violet).
La numérotation devrait être 1.1.2.
\end{theorem}

% --- Test de la Propriété ---
\begin{property}[{Test de Propriété}]
Ceci est une propriété (encadré bleu).
La numérotation devrait être 1.1.3.
\begin{itemize}
    \item Test
    \item Test 2 \\
    cocou
    \begin{example}[]
        Test d'un exemple dans une propriété
    \end{example}
\end{itemize}
\end{property}

\subsection{Test des Textes Simples}

% --- Test de la Proposition ---
% Note: Pas de titre optionnel ici pour tester
\begin{proposition}
Ceci est une proposition (texte violet).
La numérotation devrait être 1.2.1.
\end{proposition}

% --- Test du Corollaire ---
\begin{corollary}[{Test de Corollaire}]
Ceci est un corollaire (texte violet).
La numérotation devrait être 1.2.2.
\end{corollary}

% --- Test de la Remarque ---
\begin{remark}
Ceci est une remarque (texte bleu).
La numérotation devrait être 1.2.3.
\end{remark}

\subsection{Test des Exemples (Non numérotés)}

% --- Test Exemple Unique ---
\begin{example}
Ceci est un exemple unique. Le titre doit être "Ex:".
\end{example}

% --- Test Exemples Multiples ---
\begin{example}[1]
Ceci est le premier exemple d'une série. Le titre doit être "Ex 1:".
\end{example}

\begin{example}[2]
Ceci est le second exemple d'une série. Le titre doit être "Ex 2:".
\end{example}

\begin{definition}[Titre de test]
Ceci est une autre définition pour vérifier la continuité de la numérotation. La numérotation devrait être 1.2.4.
\end{definition}