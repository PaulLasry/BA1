On sait que si $f: [a, b]\overset{\text{ continue }}{\longrightarrow} \R$ alors l'intégrale est l'air sous la courbe. On généralise à $f : ]a,b [\to \R$, $f : ]-\infty, +\infty[ \to \R$.
\begin{example}
    On pourrait s'intérresser à $\int_0^1 \log(x)dx$ (qui a une asymptote verticale en $x=0$). Ou encors $\int_0^{+\infty} e^{-x}dx$.
\end{example}
\textbf{Problème : } L'approximation est toujours $\pm\infty$.\\
\textbf{Solution : } On restreint à un intervalle fermé, puis on utilise les limites. 
\begin{definition}[Intégrales généralisées]
    \begin{enumerate}
        \item Si $f : [a, b[ \to \R$ continue, $b \in \R \cup \{+\infty\}$, alors :
            \[\int_{a}^{b^-} f(x)dx = \lim_{u \to b^-}\int_{a}^{u}f(x)dx\]
        \item Si $f : ]a, b] \to \R$ continue, $a \in \R \cup \{-\infty\}$, alors :
            \[\int_{a^+}^{b} f(x)dx = \lim_{u \to a^+}\int_{u}^{b} f(x)dx\]
        \item Si $f : ]a, b[ \to \R$ continue, alors :
            \[\int_{a^+}^{b^-} f(x)dx = \lim_{u \to a^+} \int_{u}^{w} f(x)dx + \lim_{v \to b^-}\int_{w}^{u} f(x)dx\]  
            où $w \in ]a, b[$ est quelconque.
    
    \end{enumerate}
\end{definition}
\begin{remark}
    \begin{itemize}
        \item Ce sont les intégrales généralisées (ou impropres) de $f$.
        \item On dit que l'intégrale converge si la (les !) limites existe(nt) ($\in\R$), et l'intégrale diverge sinon.
        \item Pour le point 3), on peut montrer que le résultat ne dépend pas du $w$ choisi.
    \end{itemize}

\end{remark}
\textbf{Notation : } \[\int_{a}^{+\infty^-} = \int_{a}^{+\infty} \quad, \quad \int_{-\infty^+}^{b} = \int_{-\infty}^{b}\]
\begin{example}[1]
    \begin{align*}
        \int_{0^+}^{1}\log(x)dx &= \lim_{u\to 0^+} \int_{u}^{1}\log(x)dx=\lim_{u\to 0^+} \Big[x\log(x)-x\Big]_{x = u}^{x = 1}\\
        &= \lim_{x \to 0^+}(-1-u\log(u) + u) = -1+\lim_{u\to 0^+}\frac{-\log(x)}{\frac{1}{u}}\\
        &\overset{\text{B-H}}{=} -1 + \lim_{u\to 0^+}\frac{\frac{1}{u}}{-\frac{1}{u^2}} = -1
    \end{align*}
\end{example}
\begin{example}[2]
    \begin{align*}
        \int_{0}^{+\infty}e^{-x}dx &= \lim_{u\to +\infty} \int_{0}^{u} e^{-x}dx = \lim_{u\to +\infty} \Big[-e^{-x}\Big]_0^u\\
        &= \lim_{uti+\infty}(1-e^{-u}) = 1
    \end{align*}
\end{example}
\begin{example}[3]
    Pour $r > 0$, on a $\displaystyle \int_{0^+}^{1}\frac{1}{x^r}dx = \begin{cases}
        \frac{1}{1- r} & r < 1\\
        +\infty & r \geq 1
    \end{cases}$. Aussi $\displaystyle \int_{1}^{+\infty}\frac{1}{x^r}dx = \begin{cases}
        \frac{1}{1- r} & r > 1\\
        +\infty & r \leq 1
    \end{cases}$. Exemple :
    \begin{align*}
        I = \int_{0^+}^{1}\frac{1}{x^r}dx &= \lim_{u\to 0^+}\int_{u}^{1}\frac{1}{x^r}dx = \lim_{u\to 0^+}\begin{cases}
            (-\log(u)) = +\infty & r = 1\\
            \Big[\frac{x^{1-r}}{1-r}\Big]_{x = u}^{x = 1} & r\neq 1
        \end{cases}\\
        \intertext{Calculons $\Big[\frac{x^{1-r}}{1-r}\Big]_{x = u}^{x = 1}$ :}
        &\Big[\frac{x^{1-r}}{1-r}\Big]_{x = u}^{x = 1} = \frac{1}{1-r} + \frac{1}{r-1}\lim_{u\to 0^+}u^{1-r}
        \intertext{Et ceci diverge si $r > 1$ et converge vers $\frac{1}{1-r}$ si $r<1$}
        &\implies I = \begin{cases}
           \frac{1}{1- r} & r < 1\\
            +\infty & r \geq 1 
        \end{cases}
    \end{align*}
\end{example}
\begin{example}[4]
    On étudie :$\displaystyle\int_{-\infty}^{+\infty}\frac{1}{1+x^2}dx$. On coupe en $w = 0$ : 
    \begin{align*}
        \int_{-\infty}^{+\infty}\frac{1}{1+x^2}dx &=\lim_{u\to -\infty}\Big[\arctan(x)\Big]_{x = u}^0 + \lim_{u\to +\infty}\Big[\arctan(x)\Big]^{x = u}_0\\
        &= \pi
    \end{align*}
\end{example}
\begin{remark}
    Si $\int_{-\infty}^{+\infty}f(x)dx$ converge, i.e. si les deux limites existent, alors cette inégrae vaut : \[\underbrace{\lim_{u\to + \infty}\int_{-u}^{u} f(x)dx}_{\textstyle\text{veleur de cauchy}}\]
\end{remark}
\begin{example}[5]
    \begin{align*}
        \int_{-\infty}^{+\infty} xdx &= \int_{-\infty}^{0} xdx + \int_{0}^{+\infty}xdx\\
        &= \lim_{u\to -\infty}\frac{-u^2}{2} + \lim_{v\to +\infty}\frac{v^2}{2}
    \end{align*}
    L'intégrale diverge ! En revenche, sa valeur principale de Cauchy est $\lim_{u\to +\infty}\int_{-u}^{u}xdx = 0$. On voit alors que la valeur principale de Cauchy $\neq \int_{-\infty}^{+\infty}$.
\end{example}
\begin{property}[Comparaison d'intégrales]
    Soient $f, g : \left[a, b\right[\to\R$ continues telles que $0\leq f(x) \leq g(x)$ pour tout $x \in [a, b[$. Alors : 
    \begin{enumerate}
        \item $\label{prop:comp_int_1}\int_{a}^{b^-} f(x)dx$ converge $\impliedby \int_{a}^{b ^-}g(x)dx$ converge.
            \begin{proof}
                Théorème du gendarme seul!
            \end{proof}
    \end{enumerate}
\end{property}
\begin{example}[du \ref{prop:comp_int_1}]
    $I=\displaystyle\int_{0}^{1^-}\frac{1}{\sqrt{1-t^2}}dt$ converge : 
    \begin{align*}
        0 &\leq \frac{1}{\sqrt{1-t^2}}dt \leq \frac{1}{\sqrt{1-t}}
        \intertext{Et on a :}
        \int_{0}^{1^-} \frac{1}{\sqrt{1-t}}dt &\overset{x = 1-t}{=} -\int_{1}^{0^+}\frac{1}{\sqrt{x}}dx\\
        &= \int_{0^+}^{1}\frac{1}{\sqrt{x}}dx \\
        &= \left(2\sqrt{1}\right) - \lim_{u\to 0^+}\left(2\sqrt{u}\right) = 2-0 = 2
    \end{align*}
    qui converge. Donc $I$ converge par comparaison.
\end{example}
\begin{property}[Comparaison intégrales / séries]
    Soit $f: [n_0, +\infty[ \to \R$ une fonction continue, positive, et décroissante. Alors :
    \[\sum_{n = n_0}^{\infty}f(n) \text{ converge } \iff \int_{n_ 0}^{+\infty}f(x)dx \text{ converge}\]
    \textcolor{red}{\faExclamationTriangle}\textcolor{red}{Valeurs pas égales ! }
\end{property}
\begin{example}
    La somme $\displaystyle\sum_{n = 1}^{\infty}\frac{1}{n^p}$ converge $\displaystyle\iff \int_{1}^{+\infty} \frac{1}{n^p}dx$ converge si et seulement si $p>1$.
\end{example}
\begin{example}
    \begin{align*}
        \sum_{n =2}^{\infty} \frac{1}{n(\underbrace{\log(n)}_{u})^p} \; \text{ converge } &\iff \int_{2}^{+\infty}\frac{1}{x\left(\log(x)\right)^p}dx \; \text{ converge}
        \intertext{Voyons la veleur de l'intégrale indéfinie :}
        \int\frac{1}{x\left(\log(x)\right)^p}dx &\overset{du = \frac{1}{x}dx}{=} \int\frac{1}{u^p}du = \begin{cases}
            \frac{u^{1-p}}{1-p} + c & p \neq 1\\
            \log(u)+c & p = 1
        \end{cases} \\
        &=\begin{cases}
            \frac{log(x)^{1-p}}{1-p} & p\neq 1\\
            \log(\log(x)) & p = 1
        \end{cases}\\
        &\implies I = \int_{2}^{+\infty} \frac{1}{x\left(\log(x)\right)^p}dx \overset{p = 1}{=} \lim_{u \to +\infty}\Big[\log(\log(x))\Big]_{x = 2}^u = +\infty\\
        &\implies I \overset{p\neq 1}{=} \lim_{u\to +\infty} \Bigg[\frac{\log(x)^{1-p}}{1-p}\Bigg]_{x = 2}^u =c\frac{-\lim_{u\to + \infty}\log(x)^{1-p}}{1-p} - \frac{\log(2)^{1-p}}{1-p}
        \intertext{$I$ converge si et seulement si $p>1$, ainsi }
        &\sum_{n =2}^{\infty}\frac{1}{n\left(\log(n)\right)^p} \text{ converge } \iff p > 1
        \intertext{En particulier : }
        &\sum_{n = 2}^{\infty}\frac{1}{n\log(n)} \text{ diverge}\\
        &\sum_{n = 2}^{\infty}\frac{1}{n\log(n)^2} \text{ converge}
    \end{align*}
\end{example}