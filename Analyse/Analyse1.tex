\documentclass[12pt, a4paper]{report}
% --- Inclusion du style et de la bibliographie ---
\usepackage{epfl-style}
\usepackage[backend=biber, style=apa, sorting=nyt]{biblatex}
\addbibresource{sources.bib}
\usepackage{enumitem}
\numberwithin{equation}{section}

% --- En-têtes et pieds de page ---
\usepackage{fancyhdr}
\pagestyle{fancy}
\fancyhf{} % Efface tous les champs de l'en-tête et du pied de page
\renewcommand{\headrulewidth}{0pt} % Supprime la ligne sous l'en-tête

% Redéfinition de \chaptermark pour ne pas inclure le mot "Chapitre"
\renewcommand{\chaptermark}[1]{\markboth{\thechapter.\ #1}{}}

\fancyhead[L]{\itshape\nouppercase{\rightmark}} % Affiche la section actuelle en italique, à gauche
\fancyhead[R]{\itshape\nouppercase{\leftmark}}  % Affiche le numéro et titre du chapitre en italique, à droite
\fancyfoot[C]{\thepage} % Affiche le numéro de page au centre du pied de page

% Appliquer ce style aux pages de début de chapitre
\fancypagestyle{plain}{%
  \fancyhf{}%
  \fancyhead[L]{\itshape\nouppercase{\rightmark}}%
  \fancyhead[R]{\itshape\nouppercase{\leftmark}}%
  \fancyfoot[C]{\thepage}%
  \renewcommand{\headrulewidth}{0pt}%
}

% Paquet pour les icônes (le triangle attention)
\usepackage{fontawesome5}
% --- Métadonnées du Document ---
\title{Notes d'Analyse 1}
\author{Paul Lasry-Robin}
\date{\today}

% --- Document ---
\begin{document}

\maketitle % Crée la page de titre
\pagenumbering{arabic}
\begin{abstract}
  Ce document constitue une prise de notes en temps réel du cours d'Analyse \Romannum{1} d'Olivier Mila. En raison de sa nature "live", le texte peut contenir des fautes de frappe ou des imperfections de mise en page. 
  
  L'objectif de ce support est d'offrir une révision interactive et efficace grâce à l'intégration de liens internes. Si vous relevez des erreurs mathématiques, merci de me les signaler sur Telegram : \href{https://t.me/leodagan68}{\color{blue}{@leodagan68}}.
  Le code source \LaTeX\ de ce projet est disponible sur mon \href{https://github.com/PaulLasry/BA1/tree/45f9fe3004d413b7f9d8f96cf85cc4c511bb41aa/Analyse}{\color{blue}{répertoire GitHub (BA1)}}.

  Je n'ai commencé à prendre des notes qu'a partir du troisème chapitre, j'ai essayé de le completer pendant les révisions mais cela prenait trop de temps, donc le document est vide sur les chapitre des suites et séries.
\end{abstract}
\tableofcontents % Crée la table des matières

% --- Inclusion des Chapitres ---
% \include va les fichiers .tex dans le dossier 'chapitres/'
% Il force aussi un saut de page avant chaque chapitre
\graphicspath{{images/}} % Indique où chercher les images

% !TEX root = ../Analyse1.tex
\chapter{Les nombres}
\section{Entiers et rationnels}
\begin{definition}[Entiers naturels]
  On note \[\N = \{0, 1, 2, \ldots\}\quad, \quad \N^* = \N \setminus\{0\}\]
\end{definition}
\begin{definition}[Entiers relatifs]
  On note \[\Z = \{\ldots, -1, 0, 1, 2, \ldots\} = \N \cup -\N\]
\end{definition}
\begin{definition}[Nombres naturels]
  On note \[\Q = \left\{\frac{a}{b} \mid a, b\in\Z, b\ne0\right\}\]
\end{definition}
\begin{proposition}
  L'équation $x^2 = 2$ n'a pas de solution dans $\Q$
\end{proposition}
\textbf{\uline{Lemme} :} $a^2 \equiv 0  \mod 2 \implies a \equiv 0  \mod 2$
\begin{proof}[Preuve du lemme]
  Si $a$ est impaire, on a que $a = 2k +1$ où $k\in\Z$ donc $a^2 = 2(2k^2+2k) + 1$. Ainsi $a^2$ est impaire. Par la contraposé, on a que:
  \begin{quote}
    $a$ est pair si et seulement si $a^2$ est paire.
  \end{quote}
\end{proof}
\begin{proof}
  Supposons q'il existe une solution $x = \frac{a}{b}$ tel que $a, b\in\Z$ et $b\ne0$. On suppose la fraction irreductible. On a alors :
  \begin{align*}
    x^2 = 2 &\implies \frac{a^2}{b^2} = 2\implies a^2 = 2b^2
    \intertext{Par le lemme, on sait que $a$ est paire}
    a = 2c &\implies 2b^2 = 4c^2\\
    &\implies b^2 = 2c^2 \implies b^2 \equiv 0\mod 2
    \intertext{Par le lemme, $b$ est pair, donc :}
    x &= \frac{2c}{2d} = \frac{c}{d}
  \end{align*}
  La fraction est réductible, ce qui est en condratiction avec l'hypothèse. Il n'y a donc pas de solution dans $\Q$.
\end{proof}
\section{Construction des nombres réels}
On utilise la relation d'ordre sur $\Q$ pour "ajouter" des nomnbres aux entiers.
\begin{definition}[Minorant / Majorant]
  Soit $A\subseteq\Q$ un ensemble non vide.
  \begin{itemize}
    \item Un \uline{majorant / minorant} de l'ensemble $A$ est un nombre $x$ tel que $x\ge(\le)\forall a\in A$
    \item S'il existe un majorant / minorant de $A$ tel que $x\in A$ alors il s'appelle le \uline{maximum / minimum} de $A$
    \item L'ensemble $A$ est majoré / minoré / borné s'il admet un majorant / minorant / les deux.
  \end{itemize}
\end{definition}
\begin{example}
  $A = \{x\in\Q\mid0\le x\le1\}$. On a que $A$ est borné, $\min A = 0, \max A = 1$.
\end{example}
\begin{definition}[Supprémum / Infinimum]
  Soit $A\subseteq\Q (\R)$. 
  \begin{itemize}
    \item Le \uline{supprémum} de $A$ est \[\sup A = \min{x\mid x \text{ est un majorant de }A}\]
    \item Le \uline{infinimum} de $A$ est \[\inf A = \{x\mid x \text{ est un minorant de }A\}\]
  \end{itemize}
\end{definition}
\begin{remark}
  Si $A$ n'a pas de minorant / majorant, on dit par convention $\sup A = +\infty$, $\inf A = -\infty$.\\
  De plus, si $\max A$, $\min A$ existent, on a :
  \begin{align*}
    \sup A = \max A\\
    \inf A = \min A
  \end{align*}
\end{remark}
\begin{remark}
  Pour un ensemble borné, on s'attend à toujours avoir un $\inf$ et $\sup$, même si $\min$ et $\max$ n'existent pas, mais c'est faux : \[D = \{x\in\Q\mid x\le x^2\}\]
  \begin{align*}
    |x|\leq \frac32 \iff \underbrace{-\frac32}_{\text{minorant de $A$}} \leq x\leq\underbrace{\frac32}_{\text{majorant de $A$}}
  \end{align*}
\end{remark} 
\begin{proposition}
  Si $x = \sup D$ existe, alors $x^2 = 2$.
\end{proposition}
\begin{proof}[Preuve proposition]
  \begin{enumerate}
    \item Supposons que $x^2 <2$. Soit $d\in D$ tel que $d=x+\frac1n$, où $n\in\N^*$ est choisi tel que $n \ge \frac{2x+1}{2-x^2}$, on a:
      \begin{align*}
        d^2 &=\left(x+\frac1n\right)^2 = x^2 + \frac{2x}{n} + \frac1{n^2}\\
        &\le x^2 + \frac{2x}n + \frac1n = x^2 + \frac{2x+1}n \le 2\\
        \iff& \frac{2x+1}{n}\le2-x^2\iff n\le\frac{2x+1}{2-x^2}
      \end{align*}
      Comme $n$ a été choisi comme tel, on a que $d^2 \le 2$. Ainsi, $d\in D$ et $d>x$, ainsi $x$ n'est pas majorant.
    
    \item Supposons que $x^2>2$ [exercice dur !]. Comme on ne peut pas avoir ni $x^2<2$ ni $x^2>2$, on a que $x^2 = 2$
  \end{enumerate}
\end{proof}
\begin{corollary}
  Le nombre $x = \sup D$ n'existe pas dans $\Q$.
\end{corollary}
\begin{remark}
  Il n'y a pas de $x\in\Q$ tel que $x^2 = 2$.
\end{remark}
On voit qu'il manque des nombres dans $\Q$, cela nous indique à en ajouter !\\
\textbf{Construction de $\R$} :\\
$\R$ s'obtient à partir de $\Q$ en ajoutant les $\sup$ et les $\inf$ de tous les ensembles bornés $A\subseteq\Q$.
\section{Propriétés de réels}
\begin{enumerate}[label=(\roman*)]
  \item $\R$ est un corp, (on a $0, 1, +, \cdot$, inverse pour $+$ et $\cdot$, distibutivité, muni d'un ordre total ($x\le y$))
  \item Les definitions de $\min$, $\max$ etc... restent les mêmes, pour des sous-ensembles $A\subseteq\R$
  \item Réussite de la construction
\end{enumerate}
\begin{theorem}
  Pour tout $A\subseteq\R$ non-vide et minoré / majoré, le nombre $\sup A$ / $\inf A$ existe toujours et est unique.
\end{theorem}
\begin{remark}
  $\displaystyle D=\{x\in\R\mid x^2\le 2\} \implies \sup D = \sqrt{2} \quad, \quad \inf A = -\sqrt{2}$
\end{remark}

\section{Représentation décimale}
Tout nombre $x\in\R$ s'écrit $x \pm d_1d_2d_3\ldots d_n, d_{n+1}$ avec $d_i \in \N \mid 0\le d_i\le 9$
\begin{theorem}
  Soit $x\in\R$. Alors $x\in\Q$ si et seulement si $x$ a une représentation décimale finie et périodique.
\end{theorem}
\textbf{Conséquence du théorème :}
\begin{itemize}
  \item La représentation de $\sqrt2$ est bel et bien infinie, non-périodique.
  \item Densité de $\Q$ dans $\R$. Pour tous $x<y\in\R$, il existe $a\in\Q$ tel que $x<a<y$
  \item Pour tout $x\in\R$, il existe $a\in\R$ tel que $|a-x|<\varepsilon$.
\end{itemize}
\section{Nombre complexe}
Dans $\R$, on a une solution de $x^2=a$ pour tout $a\le 0$. Mais pas de solution de $x^2 = -1$. On prend $\R^2 = \R \times \R$ et on le munnit de :
\begin{enumerate}
  \item L'addition : $(a, b)+(c, d) = (a+c, b+d)$
  \item La multiplication : $(a, b)\cdot(c, d)=(ac-bd,ad+bc)$
\end{enumerate}
\textbf{Remarque /Notation :} \begin{enumerate}[label = (\roman*)]
  \item $(a, 0) + (c, 0) = (a+b, 0)$, $(a, 0) \cdot (c, 0) = (ab, 0)$. On identifie l'ensemble \[\{(x, 0)\mid x\in\R\} = \R\]
  \item Le "nombre" $(0, 1)$ est intéressant : \[(0, 1)\cdot(0, 1) = (-1, 0) = -1\]On l'appelle $i = (0, 1) =$ unité imaginaire. On dit aussi que $(a, b)=a+bi$. 
\end{enumerate} 
\begin{definition}
  L'ensemble $\R^2$ munit de $+$, $\cdot$ et de ces notations / identification est le corp des nomnbres complexe, qu'on note $\C$.
\end{definition}
\begin{remark}
  Tout nomnbre complexe $z\in\C$ s'écrit $z = a+bi$, avec $a, b\in\R$. C'est la \uline{forme cartésienne} de $z$.
\end{remark}

\section{Propriétés des nomnbres complexes}
\begin{enumerate}[label=(\roman*)]
  \item $\displaystyle\Re(z) = \frac{z+\bar{z}}{2}\quad, \quad \Im(z)=\frac{z-\bar{z}}{2}$
  \item $\overline{z_1 + z_2} = \bar{z_1}+\bar{z_2}$, $\overline{z_1 \cdot z_2} = \bar{z_1}\cdot\bar{z_2}$, $\overline{z_1 / z_2} = \bar{z_1}/\bar{z_2}$
  \item \label{prop:complex_fond_3} $|z_1|^2 = z\cdot\bar{z} \implies (a+bi)(a-bi) = a^2 + b^2 = |z|^2$
  \item $|z_1\cdot z_2 = |z_1|\cdot|z_2|\implies |z_1\cdot z_2|^2\overset{\ref{prop:complex_fond_3}}{=}z_1\cdot z_2\cdot \bar{z_1\cdot z_2}=|z_1|^2\cdot |z_2|^2$
  \item Si $z\in\C^*$, alors $\displaystyle\frac1z = \frac1{|z|^2}\cdot \bar{z}$
  \item Inégalité triangulaire : \[|z_1+z_2|\le |z_1|+|z_2|\]
\end{enumerate}
\subsubsection{Trois représentations de $\C$}
\begin{enumerate}
  \item Cartésienne : $z = a+bi$, où $a, b\in\R$
  \item Forme polaire : $z =r(\cos\theta+i\sin\theta)$ où $r\in\R^*$
  \item Forme exponentielle : $z = re^{i\theta}$
\end{enumerate}
\subsubsection{Conséquences de l'exponentielle complexe}
\begin{itemize}
  \item Pour $z\in\C, \overline{e^z} = e^{\overline{z}}$. Dnc si $z=re^{i\theta}$, alors $\overline{z}=e^{-i\theta}$.
\end{itemize}
\begin{property}
  [Formule d'Euler] \[e^{i\pi}+1 = 0\]
\end{property}
\begin{property}
  [Formule de Moivre]
  \begin{equation}
    \left(\cos\theta + i \sin\theta \right)^n = \cos(n\theta) + i\sin(n\theta)
  \end{equation}
  \begin{equation}\label{prop:com_moivre}
    \implies(e^{i\theta})^n = e^{in\theta}
  \end{equation}
\end{property}
\begin{property}
  [Formule pour $\sin$ et $\cos$] \[\cos\theta = \frac{e^{i\theta} + e^{-i\theta}}{2} \quad,\quad \sin\theta=\frac{e^{i\theta}-e^{-i\theta}}{2}\]
\end{property}

\section{Calcul dans $\C$}
\subsubsection{Calcul de $(1-\sqrt{3}i)^{30}$} 
\begin{align*}
  z &= 1-\sqrt3i\\
  |z|&=\sqrt{1 + 3} = 2\\
  \arg{z} &=\arctan\frac{-\sqrt3}{1} = \arctan-\sqrt3 = \frac{-\pi}{3}\\
  \intertext{Avec ces informations, on passe en forme polaire :}
  z^{30} &= \left(2e^{-i\frac\pi3}\right)^{30} = \ldots = 2^{30} 
\end{align*}
\subsubsection{L'équation $z^n=1$, $z=er^{i\theta}$} 
Par la formule de Moivre \eqref{prop:com_moivre}, on a :
\begin{align*}
  z^n=1 &\iff r^ne^{in\theta} = 1e^{i0}
  \intertext{On a alors que $r^n = 1$ et $n\theta=0$, donc :}
\end{align*}
\[r=1\quad,\quad\theta=\frac{k2\pi}{n}\]
Les solutions sont donc $\left\{z\in\C\mid e^{i\frac{2k\pi}{n}} : k\in\Z\right\}$. Il y a bien $n$ solutions distinctes.
\subsubsection{L'équation $z^n = w$, pour tout $w\in\C$ et $n\in\Z$} 
\noindent\textbf{Etape 1 :} Trouver une solution $z_0$, appellée solution particulière.
\begin{quote}
  \begin{example}
    Si $w=se^{i\phi}$, prendre $z_0 = \sqrt[n]se^{i\frac\phi{n}}$
  \end{example}
\end{quote}
\textbf{Etape 2 :} Trouver la solution générele. On a $z^n=w=z_0^n$. Donc : \[\left(\frac{z}{z_0}\right)^n = 1 \iff \frac{z}{z_0} \in\left\{e^{i\frac{2k\pi}{n}}\mid k\in\Z\right\}\]
On trouve a nouveau $n$ solutions distinctes : $\mathbb{S} = z_0\cdot\{\text{solutions de $z^n=1$}\}$.
\begin{theorem}[Théorème fondamental de l'algèbre]
  Tout polynomes $p(z)=a_nz^n+\ldots+a_iz^i+a_0$, avec $a_i\in\C$ se factorise en : \[p(z)=(z-z_1)(z-z_2)\cdots(z-z_n)\]
\end{theorem}
\begin{corollary}
  Toute équation polynomiale $p(z)=0$ de degré $n$ possède $n$ solutions complexes en comptant les multiplicités. 
\end{corollary}
\begin{example}
  Polynome de degré $2$. 
  \begin{align*}
    p(z)&=az^2+bz+c=0\\
    z &=\frac{-b\pm\sqrt{b^2+4ac}}{2a}
    \intertext{A interpreter comme : $\pm\sqrt{b^2-4ac}$ sont les solutions de $u^2=b^2-4ac$}
    \iff z&=\frac{-b + \text{"sols de $u^2=b^2 - 4ac$"}}{2a}
  \end{align*}
\end{example}
\begin{example}
  $z^2 +2z+3 = 0$ :
  \begin{align*}
    z&=\frac{-2\pm \text{"$\sqrt{4-12}$"}}{2}\\
    \intertext{Les solution de $u^2=-8$ sont $ u_1 =2\sqrt2\cdot i, \; u_2=-2\sqrt{2}\cdot i$, donc :}
    z&=-1\pm\sqrt2\cdot i
  \end{align*}
\end{example}
\begin{remark}
  Si $p(z)$ est à coefficient réels, alors les racines non-réelles vienntn par paire complexes conjuguées.
\end{remark}

% !TEX root = ../Analyse1.tex
\chapter{Suites}
% !TEX root = ../Analyse1.tex
\chapter{Séries}
% !TEX root = ../Analyse1.tex
% (Cette ligne aide VSCode à savoir quel fichier maître compiler)

\chapter{Fonctions}
\section{Rappels}
\begin{definition}[Fonction majorées]
    Soit $f : D\to \R$ une foonction réelle. Alors $f$ est $\underline{\text{majorée}}$ sur $A \subseteq D$ si $f(A) = \{f(x) | x \in A\}$ l'est.\\
    De plus, on pose :
    $$ \sup_{x \in A} f(x) = \sup f(A) $$
    Pareil pour inf, max et min (si min et max existent).
    \begin{example}
        Pour $f(x) = (x-1)^2 +2$ et $A = \rbrack -1, 4 \lbrack$\\
        \begin{center}
            \includegraphics[width=0.8\textwidth]{majorée.png}
        \end{center}
        Donc $\forall x \in A$ :
        \begin{itemize}
            \item $\inf f(x) = \min f(x) = 2$
            \item $\sup f(x) = 11$
        \end{itemize}
        Et max n'existe pas.
    \end{example}
\end{definition}
\newpage
\section{Limites}
\begin{example}
    $f(x) = \frac{\sin x}{x}$\\
    \noindent $D(f) = \R^*$. On aimerait définir $\lim_{x \to 0} \frac{\sin (x)}{x} = 0$.\\
    Il faut deux ingréfiants pour conclure que $\lim_{x \to x_0} f(x) = l$ :
    \begin{enumerate}
        \item $f$ doit être définie "un peu autour" de $x_0$
        \item $f(x)$ doit "s'approcher" de $l$ lorsque $x \to x_0$.
    \end{enumerate}
\end{example}
\begin{definition}[Fonction définie au voisinage]
    Une fonction $f : D \to \R$ est définie au $\underline{\text{définie au voisinage}}$ de $x_0 \in \R$ s'il existe $d \in \R > 0$ t.q :
    $$\rbrack x_0 - d , x_0 \lbrack \cup \rbrack x_0, \, x_0 + d \lbrack \subset D$$
\end{definition}
\begin{example}[$\frac{\sin x}{x}$]
        Est définie au voisinage de $x_0 = 0$ même si elle n'est pas définie en $0$.
    \end{example}
\begin{definition}[Limite d'une fonction]
    Soit $x_0 \in \R$ et $f : D \to R$ def au voisinage de $x_0$. Alors, $f$ ademt $l \in \R$ pour $\underline{\text{limite}}$ lorsque $x \to x_0$.\\
    On note : $ \lim_{x \to x_0} f(x) = l$ si $\forall \varepsilon > 0, \, \exists \delta = \delta_{\epsilon} > 0$ t.q. $\forall x \in D \setminus \{x_0\}$, on a :
    $$ |x - x_0| \leq \delta \implies |f(x) - l| \leq \varepsilon $$
\end{definition}

\begin{example}[1]
    Soit $g(x) = \begin{cases}
        \frac{\sin(x)}{x} & x \not = 0 \\
        132 & x = 0
    \end{cases}$, alors $\lim_{x \to 0} g(x) \not = 132$ car on s'interresse seulement au voisinage de $0$.    
\end{example}
\begin{example}[2]
    Soit $f(x) = 5x - 1, \ x_0 = 0$. Montrons que $\lim_{x \to 2} f(x) = 9$.\\
    \begin{enumerate}
        \item $D(f) = \R \implies f$ est def dans tout voisinage de $x_0 = 2$.
        \item Soit $\varepsilon > 0$, on pose $\delta = \frac{\varepsilon}{5}$ tel que $|x-2| \leq \delta$.\\
             $$|f(x) - 9| = |5x -10 | = 5|x-2| \leq 5\delta \leq \varepsilon$$
             Comme $\varepsilon$ est arbitraire, on a montré que :
             $$\forall \varepsilon > 0, \, \exists \delta > 0 \text{ t.q. si } x \in D \setminus \{2\} \text{ et } |x - 2| \leq \delta, \text{ on a : } $$
                $$|f(x) - 9| \leq \varepsilon$$
        
    \end{enumerate}
\end{example}

\newpage
\begin{theorem}[Limites de fonction et suites]
    Soit $f : D \to \R$ def au voisinage de $x_0 \in \R$. Aloes on peut dire que $\lim_{x \to x_0} f(x) = l \iff \lim_{x \to \infty} f(a_n) = l \; \forall(a_n)_{n\in \N}$ t.q. $\lim_{n \to \infty} a_n = x_0$.\\
    $\underline{\text{Idée : }}$ $a_n \to x_0$ sont les manière de s'approcher de $x_0$. Donc $f(x) \to l$ si $(a_n) \to l$ pour toutes les façons $(a_n \to x_0)$ de s'approcher de $x_0$. 
\end{theorem}
\begin{example}[Redémonstraion de $\lim_{x \to 2} 5x - 10 = 9$]
        Soit $(a_n)_{n \in \N} \subseteq \R \setminus \{2\}$ une suite t.q. $\lim_{n\to\infty} a_n = 2$.\\
        Alors on a :
        $$\lim_{n\to \infty} (5a_n -1) = 5(\lim_{n\to \infty} a_n) -1 = 5 \cdot 2 - 1 = 9$$
        Comme la suite était arbitraire, on a montré que pour $\underline{\text{TOUTE SUITE}}$ $(a_n)$ : $$\lim_{x \to 2} 5x - 10 = 9$$   
\end{example}
\begin{corollary}
    Si on a trouvé :
    \begin{itemize}
        \item Une suite $(a_n) \subset D \setminus \{x_0\}$ t.q. $$\lim_{n \to \infty } f(a_n) \text{  n'existe pas}$$
        \item Deux suites $(a_n), (b_n) \subseteq D \setminus \{x_0\}$ t.q. $a_n \to x_0$ et $b_n \to x_0$ mais :
        $$ \lim_{n \to \infty} f(a_n) \neq \lim_{n \to \infty} f(b_n)$$
    \end{itemize}
    Alors $\lim_{x \to x_0} f(x)$ n'existe pas.
\end{corollary}
\ \
\begin{example}[Corollaire]
    Prenons $f(x) = \cos{\frac{1}{x}} \ x_0 = 0$\\
    $D(f)=\R^* \implies f$ est définie au voisinage de $x_0 = 0$.
    En effet : 
    $$a_n = \frac{1}{2n\pi} \to 0 \qquad b_n = \frac{1}{(2n+1)\pi} \to 0$$
    Mais :
    $$\lim_{n \to \infty} f(a_n) = \lim_{n \to \infty} \cos(2n\pi ) = 1$$
    $$\lim_{n \to \infty} f(b_n) = \lim_{n \to \infty} \cos((2n+1)\pi ) = -1$$
    \begin{remark}
        On aurait aussi pu considérer la suite : 
        $$c_n = \frac{1}{n\pi} \to 0$$
        $$\implies \lim_{n\to \infty} f(c_n) = \cos(\pi n) = \lim_{n \to \infty } (-1)^n \quad \text{ qui n'existe pas}$$
        Donc $\lim_{x \to 0} \cos(\frac{1}{x})$ n'existe pas.
    \end{remark}
\end{example}

\begin{property}[Limites de fonctions] 
\noindent \\
Soient $f, g : D \to \R$ définies au voisinage de $x_0 \in \R$ et telles que :
$\lim_{x \to x_0} f(x), \, \lim_{x \to x_0} g(x)$ existent.
Alors :
\begin{enumerate}
    \item Si $\lim_{x \to x_0} f(x) = l_1$ et $\lim_{x \to x_0} f(x) = l_2$, alors $l_1 = l_2$ $\qquad$ (Unicité de le limite)
    \item $\forall p, q \in \R$ on a: $$\lim_{x\to x_0} p \cdot f(x) + q \cdot g(x) = p \cdot \lim_{x \to x_0} f(x) + q \cdot \lim_{x \to x_0} g(x )$$
    \item $$\lim_{x \to x_0} (f(x)\cdot g(x)) = \lim_{x \to x_0} f(x) \cdot \lim_{x \to x_0} g(x)$$
    \item Si $f(x) \leq g(x)$ au voisinage de $x_0$ alors : $$\lim_{x \to x_0} f(x) \leq \lim_{x \to x_0} g(x)$$
    \item Si $h : D \to \R$ est t.q. $$f(x) \leq h(x) \leq g(x)$$ au voisinage de $x_0$ et si $\lim_{x \to x_0} f(x) = \lim_{x \to x_0} g(x) = l$, alors : $$\lim_{x \to x_0} h(x) = l$$ (Théorème des deux gendarmes)
\end{enumerate}
\end{property}

\section{Calculs de limites}
Concidérons pour ces exemples $u\in \R$. 
\begin{enumerate}[start=0]
    \item $\lim_{x \to x_0} c = c$ où $c$ est une constante.
        \begin{quote}
            [Soit $a_n \to u$. On a $f(a_n) = c \to c$]
        \end{quote}        
        $\lim_{x \to u} x = u \qquad$ ($f(x) = x$)
        \begin{quote}
            [Soit $a_n \to u$. On a $f(a_n) = a_n \to u$]
        \end{quote}    
    \item \textbf{Polynômes}
        Exemple (par produit) : 
        \[
            \lim_{x \to u} x^2 = \lim_{x \to u} (x \cdot x) = \left(\lim_{x \to u} x\right) \cdot \left(\lim_{x \to u} x\right) = u \cdot u = u^2
        \]
        
        Par récurrence, on montre que $\lim_{x \to u}x^n = u^n$.
        \textit{Preuve rapide :}
        \begin{quote}
            \textbf{Init. ($n=0$):} $\lim_{x \to u} 1 = 1$. \\
            \textbf{Hérédité:} $\lim_{x \to u} x^{n+1} = \lim_{x \to u}(x^n \cdot x) = \left(\lim_{x\to u} x^n\right) \cdot \left(\lim_{x \to u}x\right) = u^n \cdot u = u^{n+1}$.
        \end{quote}
        Donc pour $P(x) = a_n x^n + a_{n-1}x^{n-1} + \cdots + a_1 x + a_0$, on a :
        \[\lim_{x \to u} P(x) = P(u)\]
    \item \textbf{Fonctions rationnelles}: $f(x) = \frac{P(x)}{Q(x)}$. Si $Q(u) \neq 0$, on a :
        \[\lim_{x \to u} Q(x) = Q(u) \quad \text{ (Point 1)}\]
        Donc par la propriété 4, on a :
        \[\lim_{x \to u} f(x) = \frac{P(u)}{Q(u)} \]
        \begin{example}
            Ainsi on a :
            $$\lim_{x \to 1} \frac{x^2 -1}{x +1} = \frac{1-1}{1+1} = \frac{0}{2}$$
            Mais si on a $Q(u) = 0$, il faut faire un travail supplémentaire :
            $$\lim_{x \to -1} \frac{x^2 -1}{x +1} = \lim_{x \to - 1} = \frac{(x -1) (x+1)}{x+1} = \lim_{x \to -1} (x-1) = -2$$
        \end{example}
    \item \textbf{$\lim_{x \to 0}\frac{\sin x}{x} = 1$, $\lim_{x \to 0}\frac{\cos}{x} = 1$}. Démostration imagée.
            \begin{center}
                \includegraphics[width=0.8\textwidth]{sinx/x.png}
            \end{center}
        Le triangle bleu est plus petit ou égal au triangle orange lui-même plus petit que le rouge. Ainsi, on a:
        \begin{align*}
            &\frac{\sin x}{2} \leq \frac{x}{2} \leq \frac{\tan x}{2}\\
            \implies& \sin x \leq x \leq \frac{\sin x}{\cos x}\\
            \implies& \frac{\sin x}{x } \leq 1 \leq \frac{1}{\cos x}\\
            \implies& \frac{\sin x \cos x}{x} \leq \cos x \leq \frac{\sin x}{x}
        \end{align*}
        Finalement, comme $\cos x\in [0, 1]$ on a :
        $$\cos x \geq \cos ^2 x = 1 - \sin^2 x \geq 1 - x^2$$
        Donc $$\cos x \geq 1-x^2$$ 
        On a alors la chaîne d'inégalités :
        $$1-x^2 \leq \cos x \leq \frac{\sin x}{x} \leq 1$$
        Ceci est vallable pour $-\frac{\pi}{2} < x < 0$, car toutes les fonctions sont paires.\\
        Par le théorème des deux gendarmes, on a :
        $$\lim_{x \to 0} \frac{\sin x}{x} = 1$$
        et $$\lim_{x \to 0}\cos x = 1$$
        \begin{example}
            On peut alors voir :
            $$\lim_{x \to 0} \sin x = \lim_{x \to 0} \frac{\sin x}{x} \cdot \lim_{x \to 0}x = 1 \cdot 0 = 0$$          
        \end{example}
\end{enumerate}
\begin{property}[Limites de fonctions composées / changement de variable]
    Soient $f : A \to B, g : B \to \R$ t.q. :
    \begin{enumerate}
        \item $\lim_{x\to a} f(x) = b \in \R$
        \item $\lim_{y\to b} g(y) = c \in \R$
        \item $f(x) \neq b$ au voisinage de $a$
    \end{enumerate}
    Alors : 
    $$\lim_{x \to a} g(f(x)) = \lim_{y \to b} g(y) = c$$
\end{property}
\begin{proof}[Preuve à l'aide des suites]
    Soit $(x_n)_{n \in \N} \subset A \setminus \{a\}$ t.q. $x_n\to a$.\\
    On pose $y_n = f(x_n)$. Alors $y_n \to b$ (par 1) et $y_n \neq b$ pour $n$ assez grand (par 3) $\implies (y_n) \subset B \setminus \{b\}$ et $y_n \to b \implies g(y_n) \to c$ (par 2)\\
\end{proof}
\begin{example}[1]
    Soit $f(x)=x^{12}-1$. Alors $\lim_{x \to 1} cos(x^{12}-1)$ vérifie 1 et 2 de la propriété au voisinage de $1$ (dés que $x\neq \pm 1$).
\end{example}
\begin{example}[2]
    On a :
    $$\lim_{x \to 0}\frac{1-\cos^2x}{3x^2 + \sin^2x} = \lim_{x \to 0}\frac{\left(\frac{\sin x}{x}\right)^2}{3 + \left(\frac{\sin x}{x}\right)^2}$$
    $$ = \lim_{y\to 1} \frac{y^2}{3 + y^2} = \frac{1}{4}$$
\end{example}
Attention : La condition 3 est indispensable, regardons un cas où elle n'est pas vérifiée.
\begin{example}[3]
    $f(x)=3$ (constante) et $g(x) = \begin{cases}
        0 & \text{si } x=3 \\
        2 & \text{si } x\neq 3
    \end{cases}$ % <--- Le '\\' a été supprimé ici

    On a $\lim_{x \to 0} g(f(x)) = \lim_{x \to 0}g(3) = 0$.

    Mais : on ne peut pas utiliser la prop car f(x) = 3 dans le voisinage de $0$. Donc:
    \[ \lim_{x \to 0} g(f(x)) \neq \lim_{y \to 3}g(y) = \lim_{y \to 3} 2 \neq 0 \]
\end{example}
\begin{property}[Limites de réciproques]
    Soit $f : [a, b] \to \R$ strictement monotone\\
    Soit $u \in [a, b]$ et $v = f(u)$. Alors 
    $f([a, b]) \to Im(f)$ est bij, et si
    $f^{-1}(Im(f)) \to [a, b]$est def au vois de $v$, alors :
    $$\lim_{x \to v} f^{-1}(v) = u$$
\end{property}
\begin{corollary}  
    Pour tout $n\in \N$, $v \geq 0$, $\lim_{x \to v} \sqrt[n]{x} = \sqrt[n]{v}$
\end{corollary}
\begin{proof}[Preuve]
    On pose $f(x) = x^n$, strictement croissante sur $[a, b] \forall b \geq 0$. Ainsi :
    $$\lim_{x \to u} \sqrt[n]{x} = \lim_{x \to v} f^{-1}(x) = f^{-1}(v) = \sqrt[n]{v}$$
\end{proof}

\section{Lim à gauche/droite, limites infinies}
\begin{definition}
    Soit $f : D \to  \R$ def. dans un voisinage à gauche (resp. à droite) de $u \in \R$, c'est à dire $\lbrack u - d, u\lbrack \; \subseteq D \; \forall d > 0$ (resp. $\lbrack u, u+d\lbrack \subseteq \; D \; \forall d > 0$).\\
    Alors $f$ admet $l \in \R$ pour limite à gauche (resp. à droite) lorsque $x\to u, \forall \varepsilon > 0\; \exists \delta > 0$ t.q. $$\forall x \in D\setminus \{u\}$$
    On a $$x \in \lbrack u-\delta, u \lbrack$$ (resp. $x \in \rbrack u, u+\delta \rbrack$) $$\implies |f(x) - l| \leq \varepsilon$$
    \textbf{Notation :} limites à gauche  : $\lim_{x \to u^-}$, limite à droite : $\lim_{x \to u^+}$\\ 
\end{definition}
\begin{example}
    $f(x) = \frac{|x|}{x}$. Il faut séparer les cas. 
    $$f(x) = \begin{cases}
        \frac{x}{x} = 1 & x>0\\
        \frac{-x}{x}=-1 & x<0
    \end{cases}$$ Donc :
    $\lim_{x \uparrow 0} f(x) = \lim_{x \uparrow 0} -1 = \lim_{x\to x^-} -1 = -1$ et $\lim_{x \downarrow 0} f(x) = \lim_{x\to x^+} =1$.
\end{example}
\begin{property}[Limites gauche droite]
    Si $f$ est def au voisinage de $u$, alors :
    $$\lim_{x \to u}f(x) = l \iff \lim_{x \to u^+}f(x) = l \iff \lim_{x \to u^-}f(x) = l$$
\end{property}
\begin{remark}
    Cela montre que $\lim_{x \to 0} f(x) = \frac{|x|}{x}$ n'existe pas.
\end{remark}
\begin{definition}
    Soit $f : D \to \R$ def au voisinage de $+\infty$ (resp. $-\infty$) c'est à dire $\lbrack a, +\infty\lbrack \subseteq D$ pour un $a\in\R$ (resp. $\rbrack -\infty, a\rbrack \subseteq D$).\\
    Alors $f(x)$ admet $l\in \R$ comme limite losrque $x \to +\infty$ (resp. $x \to -\infty$) si $\forall \varepsilon > 0 \; \exists c \in \R$ t.q. $\forall x \in D$ on a :
    $$x \geq c (x \leq c) \implies |f(x) - l|\leq \varepsilon$$
    \textbf{Notation:} $\lim_{x \to +\infty} f(x) = l$ ou $f(x) \to l$
\end{definition}
\begin{example}
    $\lim_{x\to +\infty} \frac{1}{x} = 0$.
    \begin{proof}[Preuve avec epsilon]
        Soit $\varepsilon > 0$. On pose comme $c = \frac{1}{\varepsilon}$. Alors dès que $x\geq 0$ on a :
        $$|f(x) - 0| = \frac{1}{x} \leq \frac{1}{c} \leq \varepsilon$$
        Comme $\varepsilon$ était arbitraire, on a bien montré que
        $$\forall\varepsilon > 0, \, \exists c \in \R \text{ t.q. } \forall x \in D, \, x \geq c \implies |f(x) - 0| \leq \varepsilon$$
    \end{proof}
    \begin{proof}[Preuve avec les suites]
        Soit $(x_n)_{n\in \N}$ une suite t.q. $\lim_{n\to \infty} x_n = \infty$. Alors, 
        $$\lim_{n \to \infty}f(x_n) = \lim_{n \to \infty} \frac{1}{x_n} = \frac{1}{\infty} = 0$$
        Comme $(x_n)$ était arbitraire, c'est vrai pour toute suitre. On a donc montré que $\forall(x_n) \to +\infty, \lim_{n\to \infty} f(x_n) =0$
    \end{proof}
\end{example}
\begin{remark}
    $\lim_{x \to \pm \infty} f(x) = l \iff f(x)$ a une $\underline{\text{asymptote horizontale}}$ d'équation $y=l$
        \begin{center}
            \includegraphics[width=0.8\textwidth]{asymptote-horizontal.png}
        \end{center}
\end{remark}
\begin{definition}[Divergence vers l'infini]
    Soit $f : D \to \R$ def au voisinage de $u \in \R$. Alors $f(x)$ tends vers $+\infty$ (resp. $-\infty$) lorsque $x \to u$ si $\forall A \in \R \; \exists \delta > 0$ t.q. $\forall x \in D \setminus \{u\}$ on a :
    $$|x-u| \leq \delta \iff f(x) \geq A \quad \text{(resp. } f(x) \geq A \text{)}$$
    \textbf{Notation :} $\lim_{x \to u} f(x) = +\infty$ (resp. $-\infty$) ou $f(x) \to +\infty$ (resp. $-\infty$)
\end{definition}
\begin{example}
    $\lim_{x \to 0} \frac{1}{x^2} = +\infty$
    \begin{proof}[Preuve avec epsilon]
        Soit $A \in \R$. On pose $\delta = \frac{1}{\sqrt{A}}$ (ou $\delta = 1$ si $A < 0$). Alors dès que $|x - 0 | \leq \delta$, on a :
        $$f(x) = \frac{1}{x^2} \geq \frac{1}{\delta^2} = \frac{1}{\left(\frac{1}{\sqrt{A}}\right)^2} = A$$
        Comme $A$ était arbitraire, c'est bon.
    \end{proof}
\end{example}
\begin{remark}
    \noindent\\
    \begin{itemize}
        \item On peut combiner ces lumites généralisées. Par exemple :
            $$\lim_{c\downarrow 0} \frac{1}{x} = +\infty \quad \text{et} \quad \lim_{x\uparrow 0} \frac{1}{x}= -\infty$$
        \item $\lim_{x \to u^{\pm}} f(x) = \pm \infty \iff f(x)$ admet une asymptote verticale d'eq $x = u$
        \item Les propriétés algébriques, le théorème des gendarmes, les limites de composées et réciproques, ainsi que les calculs avec $+\infty$ vallable pour les suites restent vrais pour ces limites généralisées.
        \item Attention aux formes indéterminées :
            \begin{itemize}
                \item $+\infty - +\infty$
                \item $0 \cdot +\infty$
                \item $\frac{+\infty}{+\infty}$
                \item $\frac{0}{0}$
            \end{itemize}
    \end{itemize}
\end{remark}
\begin{example}
    $$\lim_{x \to +\infty} \frac{x^2 +1}{x+1} = \lim_{x \to +\infty} \frac{x^2\left(1 + \displaystyle\frac{1}{x^2}\right)}{x\left(1 + \displaystyle\frac{1}{x}\right)} = \frac{+\infty}{1} = \infty$$
\end{example}

\section{Fonctions continues}
\begin{definition}
    Soit $f:D\to \R$ def au voisinage de $u\in \R$. Alors $f$ est $\underline{\text{continues en } x=u}$ si :
    $$\lim_{x \to u} f(x)= f(u)$$  
\end{definition}
\newpage
\begin{remark}[Concéquences de la continuité]
    Cela implique trois choses 
    \begin{enumerate}
        \item $u\in D \implies f$ def au voisinage de $u$ $\underline{\text{et}}$ en $u$.
        \item la limite $\lim_{x \to u} f(x)$ existe dans $\R$
        \item tout $f(u)\in \R$
    \end{enumerate}
\end{remark}
\begin{example}[1]
    Les polynômes, les fonctions rationnelles, $\sqrt[n]{x}$, $\sin x$ (toutes les fonction trigo), $e^x$, $\log x$ etc... sont continues $\underline{\textbf{sur leur domaine}}$.
\end{example}
\begin{example}[2]
    $f(x) = \frac{x^2 +1}{x+1}$ est continues pour tout $x\in \R \setminus \{1\}$. On voit que 
    $$\lim_{x \to 2} f(x) = \frac{2^2+1}{2-1}=5 = f(2)$$
    Mais $1 \not \in D \implies f$ n'est pas continue en $x=1$.
\end{example}
\begin{remark}
    Si $f$ est continue en $u\in \R$ et si $a_n \to u$, alors:
    $$\lim_{n\to \infty} f(a_n) = f(\lim_{n \to \infty} a_n) = f(u)$$
\end{remark}
\begin{definition}
    Soit $f:D\to \R$ def au voisinage $\begin{aligned}
        &\text{à droite }\\
        &\text{à gauche}
    \end{aligned}$ de $u\in \R$. Alors $f$ est $\textbf{continue $\begin{aligned}
        &\text{à droite }\\
        &\text{à gauche}
    \end{aligned}$ en } x = u$ si :
    \[\begin{aligned} 
        &\lim_{x \uparrow u} f(x) = f(u)\\ 
        &\lim_{x \downarrow u} f(x) = f(u)
    \end{aligned}\]
\end{definition}
\begin{example}
    $f(x) = \begin{cases}
        2x + 1 & x\geq 0 \\
        \frac{\sin(x)}{x} & x<0
    \end{cases} \implies f$ est continue en tout $x \neq 0$. En $x =0$ on a :
    $$\lim_{x \uparrow 0} f(x) = \lim_{x \uparrow 0} \frac{\sin(x)}{x} =1$$
    $$\lim_{x \downarrow 0} f(x)  \overset{x>0}{=} \lim_{x \downarrow 0} (2x+1) = 1 = f(0)$$
    Donc $f$ est continues à gauche et à droite, donc continue en $x=0$ donc continue sur $\R$ 
\end{example}
\begin{property}[Opération sur les fonctions continues]
    Si $f$ et $g$ sont continues en $u$ alors $f+g$, $\,f\cdot g$, $\,\alpha f + \beta g$, $\,\frac{f}{g}$ (si $g(u) \neq 0$) sont aussi continue.\\
    De plus, si $f$ est continue en $u$ et $g$ continue en $f(u)$, alors $(f\circ g)(x)$ est continues en $u$.
\end{property}
\begin{example}
    $\displaystyle\frac{\sin(x^2 +8x+1)}{\sqrt{x^2 + 5+\cos(x)}}$ est continue sur tout $\R$
\end{example}
\begin{definition}[Prolongement par continuité]
    Si $f : D \to \R$ est def au voisinage de $u$ avec $u\not\in \R$ et tq $\lim_{x \to u}f(x)=l\in\R$ alors, le $\underline{\text{prolongement par continuité}}$
    de $f$ est :
    \[
    \begin{aligned}
        \hat{f} : D \cup \{u\} &\longrightarrow \R\\
        x &\longmapsto 
        % Voici la partie corrigée :
        \left\{ \begin{array}{cl}
            f(x) & \text{si } x \neq u \\
            l    & \text{si } x = u
        \end{array} \right.
    \end{aligned}
    \]
\end{definition}
\begin{example}[1]
    $\displaystyle f(x) = \frac{\sin(x)}{x}$ 
    \[\hat{f} = \begin{cases}
        \frac{\sin(x)}{x} & x\neq0\\
        1 & x = 0
    \end{cases}\]
    On appelle cette fonction $\operatorname{sinc}(x)$

\end{example}
\begin{example}[2]
    A l'inverse $\cos\left(\frac{1}{x}\right)$ ne peut pas être prolongée par continuité en \(x=0\) car $\displaystyle\lim_{x \to 0}\textstyle\cos\left(\frac{1}{x}\right)$ n'existe pas.
\end{example}
\begin{definition}[Fonction continues sur un intervalle]
    Une fonction $f: [a, b] \to\R$ est $\underline{\text{continue}}$ (jusqu'au bord) si :
    \begin{enumerate}
        \item $\lim_{x\to u} f(x)=f(u) \, \forall u\in[a, b]$
        \item $\lim_{x \to a^+} f(x) = f(a)$ ($f$ est continue à droite en $x=b$)
        \item $\lim_{x\to b^-} f(x) = f(b) $ ($f$ est continue à gauche en $x=b$)

    \end{enumerate}
    De manière analogue:
    \begin{align*}
        f:& [a, b[ \to \R \text{ est continue } && \text{si 1 + 2}\\
        &]a,b] \to \R \text{ est continue } && \text{si 1 + 3}\\
        &]a,b[ \to \R \text{ est continue } && \text{si 1}
    \end{align*}
\end{definition}
\begin{theorem}[Valeur moyenne -- TVI]
    Soit $f : [a, b] \to \R$ continue. Alors:
    \[f([a, b]) = \left[\inf_{x \in [a,b]} f(x), \sup_{x \in [a, b]} f(x)\right]\]
\end{theorem}
\begin{remark}
    Cela veut dire que $f$ atteint :
    \begin{itemize}
        \item Son $\inf$ est son minimum:\[\inf_{x \in [a,b]} f(x) = \min_{x \in [a, b]} f(x) \in \R\]
        \item Son $\sup$ est son maximum:\[\sup_{x \in [a,b]} f(x) = \max_{x \in [a, b]} f(x) \in \R\]
        \item Toutes les valeurs intermédiaires.
    \end{itemize}
    Le $\min$ et $\max$ n'est pas $\pm \infty$. De plus, $f([a, b])$ est un intervalle fermé.
\end{remark}
\begin{proof}[Preuve que $f$ atteint]
    Posons $s = \sup_{x \in [a, b]}f(x) = \sup(f(\left[a, b\right]))$. On sait qu'il existe une suite $(y_n) \in f([a, b])$ tel que $y_n \to s$. Ainsi
    \begin{align*}
        &f(x_n) &&x_n \in [a, b]\\
        \implies & \exists (x_{n_k}) \text{ une sous suite de } (x_n) \text{ t.q. } x_{n_k} \to u \in [a, b]\\
        \implies & f(u) = f(\lim_{k \to \infty} x_{n_k})\\
        =& \lim_{k \to \infty} f(x_{n_k}) \\ 
        =& \lim_{k \to \infty} y_{n_k} = s
    \end{align*}
\end{proof}
\begin{example}
    L'equation $\cos(x)=x$ possède une solution $x_0 \in ]0, \frac{\pi}{2}[$.\\
    On pose \(\begin{aligned}[t]
        f: \left[0, \frac{\pi}{2}\right] &\longrightarrow  \R\\
        x &\longmapsto \cos(x) - x
    \end{aligned}\) qui est continue. On a $f(0)=\cos(0)-0=1\, \underline{ > 0}$ et $f(\frac{\pi}{2}) = \cos(\frac{\pi}{2}) - \frac{\pi}{2} = -\frac{\pi}{2} \, \underline{< 0}$.\\
    Par le théorème des valeurs intermédiaires, 
    \[f\left(\left[0, \frac{\pi}{2}\right]\right) = \left[\underbrace{\min f(x)}_{x < 0},\; \underbrace{\min f(x)}_{x > 0}\right]\] 
    Ainsi, il existe $x_0 \in \left[0, \frac{\pi}{2}\right]$ tel que $f(x_0)=0$ $\iff \cos(x_0) = x_0$ 
\end{example}
\begin{corollary}[TVI -- 1]
    Si $f: [a, b] \to \R$ est continues et que $f(a)<0$ et $f(b)>0$ (ou l'inverse), alors il existe $u\in ]a, b[$ tel que $f(u) = 0$
\end{corollary}
\begin{corollary}[TVI -- 2]
    Si $f: I\to \R$ et continue, où $I$ est un intervalle, alors $\Im(f) = f(I)$ est aussi un intervalle.
\end{corollary}
\begin{corollary}[TVI -- 3]
    Soit $f: [a, b]\to \R$ continue. Alors $f$ est injective $\iff f$ est strictement monotone. 
\end{corollary}
\begin{proof}[Preuve du Corollaire 3]
    $\underline{\impliedby}$ cf. Chap 0.\\
    $\underline{\implies}$ Supposons que $f$ n'est pas strictement monotone :
    \[\exists u < v < w \text{ t.q. } f(u) < f(v) > f(w)\]
    \begin{center}
        \includegraphics[width=0.5\textwidth]{TVI.png}
    \end{center}
    Ainsi, $f(x_1) = y = f(x_2)$, ce n'est donc pas injectif.
\end{proof}

% !TEX root = ../Analyse1.tex
\chapter{Dériviées}
\section{Définitions et exemples}
\begin{definition}[Dériviée]
    Soit $f: D\to\R$ définie au voisinage de $x_0$ ou en $x_0$. Alors $f$ est $\underline{\text{dérivable}}$  ou $\underline{\text{différentiable}}$ en $x_0$ si
    la limite
    \[f'(x_0)=\lim_{h\to 0}\frac{f(x_0+h) -f(x_0)}{h}\]
    existe ($\in \R$).\\
    \textbf{Notation :} \[f'(x_0) = \frac{df}{dx}(x_0) = \partial_xf(x_0) =\mathcal{D}_x f(x_0) = \dot{f}(x_0)\]
    On dit :\begin{itemize}
        \item $f'(x_0)$ est la dérivée de $f$ en $x_0$\\
        \item $f:D\to\R$ est $\underline{\text{dérivable}}$ si elle est dérivable en tout $x_0 \in D$.
    \end{itemize}
\end{definition}
\begin{remark}
    Le nombre $f'(x_0)$ est la pentes de la tengente à la courbe $y=f(x)$ au point$(x_0, f(x_0))$.
\end{remark}
\begin{example}
    \begin{align*}
        f'(x_0)&=\lim_{h\to 0}\frac{f(x_0+h) -f(x_0)}{h}\\
        &=\lim_{x\to x_0} \frac{f(x)-f(x_0)}{x-x_0}
    \end{align*}
\end{example}
\begin{definition}[La fonction dérivée]
    La $\underline{\text{fonction dérivée }}$ d'une fonction $f:D\to\R$ est la fonction \(\begin{aligned}[t]
        f':D(f(x))&\longrightarrow\R\\
        x&\longmapsto f'(x)
    \end{aligned}\) où $D(f') = \left\{x\in D | f\text{ est dérivable en } x\right\}$
\end{definition}
\begin{example}[1]
    $f(x) = x^2$. On a \begin{align*}
        f'(x_0) &=\lim_{h\to 0}\frac{f(x_0+h) -f(x_0)}{h}\\
        &=\lim_{h\to 0}\frac{(x_0+h)^2 -x_0^2}{h}\\
        &= \lim_{h\to 0}(2x_0 +h) = 2\cdot x_0.
    \end{align*}
    Ainsi, $f(x)=x^2$ est dérivable pour tout $x_0\in\R$. Sa dérivée est $f'(x) = 2x$.
\end{example}
\begin{example}[2]
    $f(x)=\sin(x)$, $x_0\in \R$. On a\begin{align*}
        f'(x_0) =& \lim_{h\to 0}\frac{\sin(x_0+h)-\sin(x_0)}{h}\\ 
        =&\lim_{h\to 0}\frac{\sin(x_0)\cos(h)+\cos(x_0)\sin(h)-\sin(x_0)}{h}\\ 
        =& \sin(x_0)\cdot\lim_{h\to 0} \frac{\cos(h)-1}{h} + \cos(x_0)\cdot\lim_{h\to 0}\frac{\sin(h)}{h}\\ 
        \implies& \underbrace{-h}_{\to 0} = \frac{1-h^2-1}{h}\geq\underbrace{\frac{\cos(h)-1}{h}}_{\to 0}\geq \frac{0}{h} = \underbrace{0}_{\to 0}\\
        \implies& \sin(x_0)\cdot 0+\cos(x_0)\cdot1 = \cos(x_0)
    \end{align*}
    $\sin$ est dérivable sur $\R$ et $\sin'(x) = \cos(x)$. De manière analogue : $\cos'(x) = -\sin(x)$²
\end{example}
\begin{property}
    Soit $f:D\to\R$
    \begin{enumerate}
        \item Si $f$ est dérivable en $x_0$, alors $f$ est aussi continue en $x_0$
        \item $f$ est dérivable en $x_0$ si et seulement si :\[f(x) = \underbrace{f(x_0)+f'(x_0)\cdot(x-x_0)}_{\text{équation de la tengente}}+ \underbrace{\textcolor{red}{(x-x_0)\cdot\varepsilon(x)}}_{\text{reste}}\]
            où $\varepsilon(x)$ est une fonction tel que $\lim_{x\to x_0} \varepsilon(x)=0$. Le \textcolor{red}{reste} tend plus vite vers $0$ que $x-x_0$
    \end{enumerate}
\end{property}

% !TEX root = ../Analyse1.tex
\chapter{Intégrales}
\section{Primitives et intégrales }
\begin{definition}[Primitive]
    Soit $f:I\to\R$ une fonction continue où $I$ est intervalle. Une \underline{primitive} de $f$ est une fonction $F:I\to\R$ telle que \[F'(x)=f(x), \quad \forall x\in I.\]
\end{definition}
\begin{remark}
    Si $F, G$ sont deux primitives de la même fonction $f$, alors :
    \[F(x) - G(x) = f(x) - f(x) = 0\]
    Donc $F$ et $G$ diffèrent d'une constante : il existe $c\in\R$ tel que \[F(x) = G(x) + c, \quad \forall x\in I.\]
\end{remark}
\textbf{Notation :} $\int f(x) dx$ est l'ensemble de \underline{toutes} les primitives de $f$. Donc :
\[\int f(x)dx = \{F(x)+c | c\in\R\}\] où $F$ est une primitive de $f$. Alors la notation \[\int f(x) dx = F(x)+c\]
\begin{remark}
    L'intégrale $\int f(x)dx$ s'appelle \underline{intégrale indéfinie} de $f$.
\end{remark}
Changeons de point de vue, pour $f : [a,b]\to\R$.
\begin{center}
    \includegraphics[width=0.8\textwidth]{int_approx.png}
\end{center}
Approx 1 :
\[\text{Aire } \gtrsim  \sum_{i = 1}^{n} (x_i - x_{i-1}) \cdot (\inf_{x\in[x_{i-1}, x_i]} f(x))\]
Approx 2 :
\[\text{Aire } \lesssim  \sum_{i = 1}^{n} (x_i - x_{i-1}) \cdot (\sup_{x\in[x_{i-1}, x_i]} f(x))\]
\begin{definition}[Intégrale de Riemann]
    Soit $f:[a,b]\to\R$ est \underline{intégrale} (au sens de Riemann) si \[\sup\{\text{ Approx 1 }\} = \inf\{\text{ Approx 2 }\} = A \in\R\]
    Dans ce cas, on note \[\int_a^b f(x) dx = A\]
    Dans ce cas, on parle d'intégrale définie de $f$ entre $a$ et $b$.
\end{definition}
\begin{remark}
    On dit que $\int_a^b f(x)dx$ donne l'aire signée sous la courbe. Par convention, on dit que $\int_a^a = 0$ et que $\int_b^a f(x) dx = -\int_a^b f(x) dx$.
\end{remark}
\begin{theorem}
    Si $f : [a,b]\to\R$ est continue, monotone, bornée et continue partout sauf en un nombre finis de points, alors $f$ est intégrable (au sens de Riemann) sur $[a,b]$.
\end{theorem}
\begin{proof}[Preuve]
    Technique (Ex : monotone)
\end{proof}
\begin{example}[Contre exemple]
    Soit $f : [0,1]\to\R$ définie par \[f(x) = \begin{cases}
        1 & x\in\Q\\
        0 & x\notin\Q
    \end{cases}\]
    Alors $f$ n'est pas intégrable sur $[0,1]$. En effet, pour toute subdivision $0 = x_0 < x_1 < \ldots < x_n = 1$, on a
    \[\text{Approx 1} = \sum_{i=1}^n (x_i - x_{i-1}) \cdot 0 = 0\]
    car dans chaque intervalle $[x_{i-1}, x_i]$ il existe des réels irrationnels. De même,
    \[\text{Approx 2} = \sum_{i=1}^n (x_i - x_{i-1}) \cdot 1 = 1\]
    car dans chaque intervalle $[x_{i-1}, x_i]$ il existe des rationnels. Donc \[\sup\{\text{Approx 1}\} = 0 \neq 1 = \inf\{\text{Approx 2}\}\] et $f$ n'est pas intégrable au sens de Riemann.
\end{example}
\begin{remark}
    Cette fonction est intégrable au sens de Lebesgue mais ce n'est pas le sujet de ce cours.
\end{remark}
\begin{property}[Intégrale de Riemann]
    Les fonctions intégrables au sens de Riemann ont ces propriétés. Soient $f, g : [a,b]\to\R$ intégrables et $\alpha, \beta \in\R$. Alors :
    \begin{enumerate}
        \item $\displaystyle\int_{a}^{b}(\alpha f(x) + \beta g(x)) dx = \alpha \int_a^b f(x) dx + \beta \int_a^b g(x) dx$ (linéarité)
        \item Si $a <u <b$, alors : $\displaystyle\int_a^b f(x) dx = \int_a^u f(x) dx + \int_u^b f(x) dx$ (additivité) 
        \item Si $f(x)\leq g(x), \forall x\in[a,b]$, alors $\displaystyle\int_a^b f(x) dx \leq \int_a^b g(x) dx$ (monotonie)
    \end{enumerate}
\end{property}
\begin{proof}[Preuve]
    Technique, mais :
    \begin{enumerate}
        \item Linéarité : \[\sup_{x\in [x_{i-1}, x_i]} (\alpha f(x) + \beta g(x)) \leq \alpha \sup_{x\in [x_{i-1}, x_i]} f(x) + \beta \sup_{x\in [x_{i-1}, x_i]} g(x)\]
    \end{enumerate}
\end{proof}
\begin{remark}
    On peut écrire l'intégrale avec n'importe quelle variable : \[\int_{a}^{b} f(x)dx = \int_{a}^{b}f(y)dy\]
    On a aussi :\[-\left|f(x)\right|\leq f(x)\leq\left|f(x)\right|\implies \left|\int_{a}^{b} f(x)dx\right| \leq \int_{a}^{b}\left|f(x)\right|dx\]
\end{remark}
\begin{theorem}[Théorème de la moyenne]
    Soit $f:[a, b]\to\R$ continue. Alors il existe $u\in]a, b[$ tel que : \[\int_{a}^{b}f(x)dx = f(u)\cdot(b-a)\]
\end{theorem}
\begin{proof}
    On prend $m = \min_{x\in[a, b]} f(x)\leq f(x) \leq M = \max_{x\in[a, b]} f(x)$. Donc :
    \[m(b-a) \leq \int_{a}^{b} f(x) dx \leq M(b-a)\]
    Ainsi : \[m \leq \frac{1}{b-a} \int_{a}^{b}f(x)dx\leq M\]
    Par le théorème des valeurs intermédiaires, il existe $u\in]a, b[$ tel que $f(u) = \frac{1}{b-a} \int_{a}^{b}f(x)dx$.
\end{proof}
\begin{remark}
    Donc $f(u)$ est la valeur moyenne de $f$ sur $[a, b]$.
\end{remark}
\begin{theorem}[Théorème fondamental du calcul intégrale]
    Soit $f : [a, b]\to \R$ continue. Alors 
    \begin{enumerate}
        \item La fonction $G : [a, b]\to \R$ définie par \[G(x) = \int_{a}^{x} f(t) dt\] est une primitive de $f$ sur $[a, b]$.
        \item Si $F$ est une primitive de $f$ sur l'intervalle $[a, b]$, alors : \[\int_{a}^{b}f(x)dx = F(b)-F(a)\]
    \end{enumerate}
\end{theorem}
\begin{proof}
    
    \textbf{Parie 1 :} On dérive $G$ : \begin{align*}
        G'(x) &= \lim_{h\to0}\frac{G(x+h)-G(x)}{h} = \lim_{h\to 0}\frac{1}{h}\left(\int_{a}^{x+h}f(t)dt - \int_{a}^{x} f(t)dt\right)\\
        &= \lim_{h\to 0}\frac{1}{h} \underbrace{\int_{x}^{x+h}f(t)dt}_{f(u)h \text{ thm de la moyenne}}\\
        &= \lim_{h\to 0} \frac{1}{h}\underbrace{f(u)h}_{\in]x, x+h[} = \lim_{u\to x}f(x)
    \end{align*}
    \textbf{Partie 2 :} On a $F(x) = G(x) + c$, donc :
    \begin{align*}
        F(b) - F(a) &= (G(b) + c) - (G(a) + c)\\
        &= G(b) - G(a) = \int_{a}^{b} f(t) dt - \int_{a}^{a}f(t)dt\\
        &= \int_{a}^{b} f(t) dt
    \end{align*} 
\end{proof}
\textbf{Notation : } On note $F(x)\Big|_{a}^{b} = \Big[F(x)\Big]_a^b= F(b) - F(a)$.

\section{Calculs d'intégrale}
\begin{example}[Facile]
    \noindent On calcule.
    \begin{enumerate}
        \item $\displaystyle\int_0^\pi \sin(x)dx = \Big[-\cos(x)\Big]_0^\pi = -\cos(\pi) + \cos(0) = 2$. Mais attention : aire signée ! Donc $ \displaystyle\int_0^{2\pi} \sin(x)dx = \Big[-\cos(x)\Big]_0^{2\pi} = -\cos(2\pi) + \cos(0) = 0$
        \item $\displaystyle\int (3x+1)dx = \frac{3}{2}x^2 + x + c$
        \item $\displaystyle\int a^x dx = \int e^{\ln(a)x} dx = \frac{1}{\ln(a)} e^{\ln(a)x} + c = \frac{a^x}{\ln(a)} + c$
        \item $\displaystyle\int f(x)f'(x)dx = \frac{1}{2}f(x)^2 + c$ (vérifier en dérivant). Ex : \[\int \sin(x)\cos(x)dx = \frac{1}{2} \sin^2(x) + c\] 
        \item $\displaystyle\int\frac{f'(x)}{f(x)}dx = \ln|f(x)| + c$ (vérifier en dérivant). Ex : \[\int \frac{-\sin(x)}{\cos(x)} dx = -\ln|\cos(x)| + c\]
        \item $\displaystyle\int_{0}^{\frac{\pi}{2}}\cos^2(x)dx = \int_{0}^{\frac{\pi}{2}}\frac{1+\cos(2x)}{2}dx = \Bigg[\frac{x}{2}+\frac{\sin(2x)}{4}\Bigg]_0^{\frac{\pi}{2}}$
    \end{enumerate}
\end{example}
\begin{property}[Changement de variable/substitution]
    Soit $f: [a, b] \to \R$ une fonction continue et $\varphi : [u, v]\to [a, b]$ une fonction de la classe $C$ telle que $\varphi (u) = a, \varphi (v) = b$. Alors :
    \[\int_{a}^{b} f(x)dx \underset{x = \varphi(t)}{=} \int_{u}^{v} f(\varphi(t)) \varphi'(t)dt\]
\end{property}
\begin{remark}
    Si $\varphi$ est bijective, alors on peut écrire :
    \begin{align*}
        F(x)=F(\varphi(\varphi^{-1}(x))) = G(\varphi^{-1}(x))
    \end{align*}
    Donc $\int f(x)dx = \int f(\varphi(t)) \varphi'(t) dt$.
\end{remark}
\begin{example}
    On pose $\int_{0}^{1}\sqrt{1 - x^2} dx$. En posant :
    \begin{align*}
        \varphi : \left[0, \frac{\pi}{2}\right] &\longrightarrow [0, 1]\\
        t &\longmapsto \sin(t)
    \end{align*}
    Cette fonction est de classe $C^1$, $\varphi(0) = 0$ et $\varphi(\frac{\pi}{2}) = 1$. Donc :
    \begin{align*}
        \int_{0}^{1}\sqrt{1-x^2}\; dx &= \int_{0}^{\frac{\pi}{2}} \sqrt{1-\sin^2(t)} \cos(t) \;dt\\
        &= \int_{0}^{\frac{\pi}{2}}\sqrt{\cos^2(t)}\cos(t) \;dt\\
        &= \int_{0}^{\frac{\pi}{2}} \cos^2(t)\;dt = \frac{\pi}{4} \quad \text{ cf exemple précédent}
    \end{align*}
\end{example}
\begin{example}[indéfinie]
    On reprend $\int\sqrt{1-x^2}$. On pose :
    \begin{align*}
        \varphi : [-\frac{\pi}{2}, \frac{\pi}{2}] &\to [-1, 1]\\
        t &\mapsto \sin(t)
    \end{align*}
    $\varphi$ est bijective. Donc : 
    \begin{align*}
        \int\sqrt{1-\sin^2(t)}\cos(t)\; dt &= \int cos^2(t)\; dt\\
        &= \frac{t}{2}+\frac{1}{4}\sin(2t)+c
    \end{align*}
    On évalue en $t = \arcsin(x)$ :
    \begin{align*}
        \int \sqrt{1-x^2} \; dx = \frac{\arcsin(x)}{2} + \frac{1}{2}x\sqrt{1-x^2}+c
    \end{align*}
\end{example}
\begin{remark}
    On peut aussi exprimer $t$ en fonction de $x$. 
\end{remark}
\begin{example}[remaeque]
    On pose $\int e^{x^2} dx$. Donc $t = x^2$, $\frac{dt}{dx} = 2x$ :
    \begin{align*}
        \int e^{x^2} dx &= \frac{1}{2}\int e^{t}\; dt = \frac{1}{2}e^t +c\\
        &= \frac{1}{2}e^{x^2} + c 
    \end{align*}
\end{example}
Comment bien choisir \underline{la} substitution ? C'est dur ! Voici quelques exemples :
\begin{itemize}
    \item $\displaystyle\int e^{x^2} dx$ : $t = x^2$
    \item $\displaystyle\int\frac{x}{1+x^2}dx, \int\frac{\sin(x)}{(1+\cos(x))^3} dx$ : $t =1+\cos(x)$, ou $t = 1+x^2$. Il faut prendre ce qu'il ya "sous" le dénominateur, ou mieux "dedans dessous".
    \item $\begin{aligned}[t]
            \displaystyle\int\sqrt{1-x^2} dx\\
            x = \sin(t)
        \end{aligned}, \begin{aligned}[t]
            \int\sqrt{1+x^2}dx\\
            x = \sinh(t)
        \end{aligned}$
    \item En cas de forces majeures, pour les fonction rationnelles en $\sin$ ou $\cos$ comme : \[\int\frac{1}{\sin(x)}dx, \int\frac{1}{\sin^4(x)}dx\] on pose 
    \begin{quote}
        $t = \tan(x)$ donc $\sin(x) = \displaystyle\frac{t}{\sqrt{1+t^2}}$ et $\cos(x) = \displaystyle\frac{1}{\sqrt{1+t^2}}$.\\
        ou $t = \tan(\frac{x}{2})$, donc $\displaystyle\sin(x)=\frac{2t}{1+t^2}, \cos(x) = \frac{1-t^2}{1+t^2}, dx = \frac{2dt}{1+t^2}$.
    \end{quote}
\end{itemize}
\begin{example}
    $\displaystyle\int \frac{1}{\sin^4(x)} dx$.
    \begin{align*}
        \int\frac{1}{\sin^4(x)}dx &\overset{t = \tan(x)}{=} \int\frac{\left(1+t^2\right)^2}{t^4} \frac{1}{1+t^2} dt\\
        &=\int t^{-4} + 2t^{-2} + 1\; dt = \frac{t^{-3}}{-3} + \frac{t^{-1}}{-1}+c\\
        &= -\frac{1}{3\tan^3(x)} - \frac{1}{\tan(x)} + c
    \end{align*}
\end{example}
\begin{example}
    $\displaystyle\int \frac{1}{\sin(x)}dx$. 
    \begin{align*}
        \int\frac{1}{\sin(x)} dx &\overset{t = \tan\left(\frac{x}{2}\right)}{=} \int \frac{1+t^2}{2t} \cdot \frac{2}{1+t^2} dt\\
        &= \int \frac{1}{t} dt = \ln|t| + c = \ln\left|\tan\left(\frac{x}{2}\right)\right| + c
    \end{align*} 
\end{example}
\begin{property}[Intégrale par parties]
    Soit $f \in C^0([a, b])$ et $g \in C^1([a, b])$ et $F$ une primitive de $f$. Alors \[\int_{a}^{b}f(x)dx = \Big[F(x)g(x)\Big]_a^b -\int_{a}^{b} F(x)g'(x)dx\]
\end{property}
\begin{proof}
    On a $\left(Fg\right)' = F'g+Fg' = fg+Fg$, donc :
    \begin{align*}
        \int_{a}^{b}f(x)g(x)dx &= \int_{a}^{b} \left(F(x)g(x)\right)' dx - \int_{a}^{b} F(x)g'(x) dx\\
        &= \Big[F(x)g(x)\Big]_a^b - \int_{a}^{b} F(x)g'(x) dx
    \end{align*}
\end{proof}
\begin{remark}
    La preuve montre que c'est pareil pour les intégrales indéfinies.
\end{remark}
\begin{example}[1]
    \[\int e^xdx = e^xx-\int e^x \cdot 1 dx = e^x(x-1)+c\]
\end{example}
\begin{example}[2]
    \begin{align*}
        \int \ln(x)dx &= \int \ln(x)\cdot 1dx \\ 
        &= \ln(x)x - \int \frac{1}{x}x dx\\
        &= x\ln(x) - x + c \\
        &= x(\ln(x) - 1) + c
    \end{align*}
\end{example}
\begin{example}[3]
    \begin{align*}
        \underbrace{\int \cos^2(x)dx}_{I} &= \int \cos(x)\cdot \cos(x) dx\\ 
        &= \sin(x)\cos(x) - \int \sin(x) \cdot \left(-\sin(x)\right)dx\\
        &= \sin(x)\cos(x) + \int \sin^2(x)dx\\
        &= \sin(x)\cos(x) + \int 1 dx - \underbrace{\int \cos^2(x)dx}_{I}\\
        &\implies 2I = \sin(x)\cos(x)+x - I\\
        &\implies 2I = \sin(x)\cos(x)+x+c\\
        &\implies I = \frac{1}{2}\left(\sin(x)\cos(x) + x\right) + c
    \end{align*}
\end{example}
\begin{example}[4]
    (Intégrale par récurrence)
    \begin{align*}
        A_n &= \int_{0}^{\frac{\pi}{2}} \cos^{2n}(x)dx\\
        \implies A_0 &=\int_{0}^{\frac{\pi}{2}}1dx = \frac{\pi}{2}\\
        \intertext{Pour  $n\geq 1$ :}
        A_n &= \int_{0}^{\frac{\pi}{2}} \cos(x)\cos^{2n-1}(x)dx\\ 
        &= \Big[\sin(x)\cos^{2n-1}(x)\Big]_0^{\frac{\pi}{2}} - \int_{0}^{\frac{\pi}{2}}\sin(x)(2n-1)\cos^{2n-2}(x)(-sin(x))dx\\
        &= (2n-1)\int_{0}^{\frac{\pi}{2}}\sin^2(x)\cos^{2(n-1)}(x)dx\\
        &= (2n-1)\int_{0}^{\frac{\pi}{2}}\cos^{2(n-1)}(x)dx - (2n-1)\int_{0}^{\frac{\pi}{2}}\cos^{2n}(x)dx\\
        \implies A_n &= (2n-1)A_{n-1} - (2n-1)A_n\\
        \implies 2nA_n &= (2n-1)A_{n-1}\\
        \implies A_n &= \frac{2n-1}{2n}A_{n-1}
    \end{align*}
\end{example}
\subsection{Intégration de fonctions rationnelles}
Les fonctions rationnelles sont des fonctions de la forme $\frac{p(x)}{q(x)}$ où $p, q$ sont des polynômes. Pour intégrer ces fonctions, on utilise la décomposition en éléments simples.
\subsubsection{Building blocks}
\begin{enumerate}[label=\roman*), ref=\text{\roman*}]
    \item $\displaystyle\int \frac{1}{x}dx = \log|x| + c$. Ainsi : \[\int \frac{1}{ax+b}dx = \int \frac{1}{u}\frac{1}{a}du = \frac{1}{a}\log|ax+b| +c\] avec $u = ax+b \implies du = a dx\implies dx = \frac{1}{a}du$.
    \item $\displaystyle\int \frac{1}{x^k}dx = \int x^{-k}dx = \frac{x^{-k+1}}{-k+1}+c = \frac{x^{1-k}}{1-k} + c$, ainsi :
        \[\int\frac{1}{\left(ax+b\right)^k}dx = \frac{1}{a}\int \frac{1}{u^k}du = \frac{1}{a}\frac{u^{1-k}}{1-k} + c = \frac{1}{a}\frac{\left(ax+b\right)^{1-k}}{1-k}+c\]
    \item \label{prop:3} $\displaystyle\int \frac{1}{x^2+1}dx = \arctan(x) + c$. Si $q(x) = ax^2 +bx+c$ est tel que $\Delta < 0$, alors on peut "completer le carré" :
        \begin{align*}
            q(x)&= a\left(\left(x+\frac{b}{2a}\right)^2 + \frac{-\Delta}{4a^2}\right)\\
            \intertext{On pose $d^2 = \frac{-\Delta}{4a^2} > 0$, donc :}
            &= ad^2\left(\left(\frac{x+\frac{b}{2na}}{d}\right)+1\right)\\
            \intertext{On pose $u = \frac{x+\frac{b}{2na}}{d} \implies du = \frac{1}{d}dx \implies dx = d\cdot du$, donc :}
            \int \frac{1}{ax^2+bx+c}dx &= \frac{1}{ad^2}\int\frac{1}{u^2+1}d\cdot du \\
            &= \frac{1}{ad}\arctan(u) + c = \frac{1}{ad}\arctan\left(\frac{x+\frac{b}{2na}}{d}\right) + c
        \end{align*}
    \item $\displaystyle\int \frac{x}{ax^2+bx+c}dx = I$. On a alors :
        \begin{align*}
            I &= \frac{1}{2a}\int\frac{2ax+b}{ax^2+bx+c}dx - \frac{b}{2a}\int \frac{1}{q(x)}dx\\
            &= \frac{1}{2a}\log|ax^2+bx+c| + (\ref{prop:3}) + c
        \end{align*}
    \item $\displaystyle\int \frac{2ax+b}{\left(ax^2+bx+c\right)^k}dx = \int \frac{1}{u ^k}du$ avec $u = ax^2+bx+c \implies du = (2ax+b)dx$. Donc :
        \[\int \frac{2ax+b}{\left(ax^2+bx+c\right)^k}dx = \frac{(ax^2+bx+c)^{-k+1}}{-k+1} + c\]
    \item $\displaystyle\int \frac{1}{\left(ax^2+bx+c\right)^k}dx$. En exercices.
\end{enumerate}
A l'aide de ces 6 building blocks, ont peut intégrer tout $f(x) = \frac{p(x)}{q(x)}$ avec le décomposition en éléments simples.
\subsubsection{Méthode 1}
\begin{enumerate}
    \item Si $\deg(p)\geq deg(q)$, faire la division polynomiale !
        \begin{example}
            Prenons $\int\frac{3x^4+6}{x^4-x^3-x+1}dx$. On fait la division polynomiale :
            \[= \int \frac{3(x^4 - x^3-x+1)}{x^4-x^3-x+1}dx + \int\frac{3x^3+3x+3}{x^3 -x^3-x+1}dx\]\[=\int 3dx + \int\frac{\ldots}{q(x)}dx = 3x + \int \frac{3x^3+3x+3}{x^4-x^3-x+1}dx\]
        \end{example}
    \item Factoriser le dénominateur $q(x)$ et décomposer. On a $q(x)=(x-u)^k\cdot(ax^2+bx+c)(x-v)$. Ainsi on peut écrire : 
        \[\frac{p(x)}{q(x)} = \frac{A}{x-u}+ \frac{A_2}{(x-u)^2}+\frac{A_k}{(x-u)^k}+\frac{Bx+C}{ax^2+bx+c}\]
        \begin{example}
            $q(x) = x^4 -x^3 -x +1 = (x-1)^2(x^2+x+1)$. Donc :
            \begin{align*}
                \frac{p(x)}{q(x)} &= \frac{3x^3+3x+3}{x^4 - x^3 - x + 1} = \frac{A_1}{x-1} + \frac{A_2}{(x-1)^2} + \frac{Bx+C}{x^2+x+1}\\
                &= \frac{(A_1+B)x^3 + (A_2-2B+C) x^2 + (A_2 + B - 2C) x + (-A_1+A_2+C)}{x^4 - x^3 -x + 1}
                \intertext{On pose alors le système :}
                \implies &\begin{cases}
                    A_1 + B = 3\\
                    A_2 - 2B + C = 0\\
                    A_2 + B - 2C = 3\\
                    -A_1 + A_2 + C = 3
                \end{cases}\\
                \iff& \begin{cases}
                    A_1 = 1\\
                    B = 2\\
                    A_2 = 3\\
                    C = 1
                \end{cases}
                \intertext{Donc :}
                &\frac{3x^3 + 3x + 3}{x^4- x^3-x+1} = \frac{1}{x-1} + \frac{3}{(x-1)^2} + \frac{2x+1}{x^2+x+1} 
            \end{align*}
        \end{example}
    \item Intégrer les éléments simples à l'aide des building blocks.
        \begin{example}
            \begin{align*}
                \int \frac{1}{x - 1}dx &= \log|x-1| + c\\
                \int \frac{3}{\left(x-1\right)^2}dx &= \int 3(x-1)^{-2}dx = \frac{-3}{x-1}+c\\
                \int \frac{2x+1}{x^2+x+1}dx &= \log|x^2+x+1| + c 
                \intertext{Ainsi :}
                \int \frac{3x^4+ 6}{x^4-x^3-x + 1}dx &= 3x + \log|x-1| - \frac{3}{x-1} + \log|x^2+x+1| + c
            \end{align*}
        \end{example}
\end{enumerate}

\section{Intégrale généralisée ou impropres}
On sait que si $f: [a, b]\overset{\text{ continue }}{\longrightarrow} \R$ alors l'intégrale est l'air sous la courbe. On généralise à $f : ]a,b [\to \R$, $f : ]-\infty, +\infty[ \to \R$.
\begin{example}
    On pourrait s'intérresser à $\int_0^1 \log(x)dx$ (qui a une asymptote verticale en $x=0$). Ou encors $\int_0^{+\infty} e^{-x}dx$.
\end{example}
\textbf{Problème : } L'approximation est toujours $\pm\infty$.\\
\textbf{Solution : } On restreint à un intervalle fermé, puis on utilise les limites. 
\begin{definition}[Intégrales généralisées]
    \begin{enumerate}
        \item Si $f : [a, b[ \to \R$ continue, $b \in \R \cup \{+\infty\}$, alors :
            \[\int_{a}^{b^-} f(x)dx = \lim_{u \to b^-}\int_{a}^{u}f(x)dx\]
        \item Si $f : ]a, b] \to \R$ continue, $a \in \R \cup \{-\infty\}$, alors :
            \[\int_{a^+}^{b} f(x)dx = \lim_{u \to a^+}\int_{u}^{b} f(x)dx\]
        \item Si $f : ]a, b[ \to \R$ continue, alors :
            \[\int_{a^+}^{b^-} f(x)dx = \lim_{u \to a^+} \int_{u}^{w} f(x)dx + \lim_{v \to b^-}\int_{w}^{u} f(x)dx\]  
            où $w \in ]a, b[$ est quelconque.
    
    \end{enumerate}
\end{definition}
\begin{remark}
    \begin{itemize}
        \item Ce sont les intégrales généralisées (ou impropres) de $f$.
        \item On dit que l'intégrale converge si la (les !) limites existe(nt) ($\in\R$), et l'intégrale diverge sinon.
        \item Pour le point 3), on peut montrer que le résultat ne dépend pas du $w$ choisi.
    \end{itemize}
    Ce sont les intégrales généralisées (ou impropres) de $f$.
\end{remark}
\textbf{Notation : } \[\int_{a}^{+\infty^-} = \int_{a}^{+\infty} \quad, \quad \int_{-\infty^+}^{b} = \int_{-\infty}^{b}\]
\begin{example}[1]
    \begin{align*}
        \int_{0^+}^{1}\log(x)dx &= \lim_{u\to 0^+} \int_{u}^{1}\log(x)dx=\lim_{u\to 0^+} \Big[x\log(x)-x\Big]_{x = u}^{x = 1}\\
        &= \lim_{x \to 0^+}(-1-u\log(u) + u) = -1+\lim_{u\to 0^+}\frac{-\log(x)}{\frac{1}{u}}\\
        &\overset{\text{B-H}}{=} -1 + \lim_{u\to 0^+}\frac{\frac{1}{u}}{-\frac{1}{u^2}} = -1
    \end{align*}
\end{example}
\begin{example}[2]
    \begin{align*}
        \int_{0}^{+\infty}e^{-x}dx &= \lim_{u\to +\infty} \int_{0}^{u} e^{-x}dx = \lim_{u\to +\infty} \Big[-e^{-x}\Big]_0^u\\
        &= \lim_{uti+\infty}(1-e^{-u}) = 1
    \end{align*}
\end{example}
\begin{example}[3]
    Pour $r > 0$, on a $\displaystyle \int_{0^+}^{1}\frac{1}{x^r}dx = \begin{cases}
        \frac{1}{1- r} & r < 1\\
        +\infty & r \geq 1
    \end{cases}$. Aussi $\displaystyle \int_{1}^{+\infty}\frac{1}{x^r}dx = \begin{cases}
        \frac{1}{1- r} & r > 1\\
        +\infty & r \leq 1
    \end{cases}$. Exemple :
    \begin{align*}
        I = \int_{0^+}^{1}\frac{1}{x^r}dx &= \lim_{u\to 0^+}\int_{u}^{1}\frac{1}{x^r}dx = \begin{cases}
            \lim_{u\to 0^+}(-\log(u)) = +\infty & r = 1\\
            \lim_{u\to 0^+} \Big[\frac{x^{1-r}}{1-r}\Big]_{x = u}^{x = 1} & r\neq 1
        \end{cases}\\
        \intertext{Calculons $\Big[\frac{x^{1-r}}{1-r}\Big]_{x = u}^{x = 1}$ :}
        &\Big[\frac{x^{1-r}}{1-r}\Big]_{x = u}^{x = 1} = \frac{1}{1-r} + \frac{1}{r-1}\lim_{u\to 0^+}u^{1-r}
        \intertext{Et ceci diverge si $r > 1$ et converge vers $\frac{1}{1-r}$ si $r<1$}
        &\implies I = \begin{cases}
           \frac{1}{1- r} & r < 1\\
            +\infty & r \geq 1 
        \end{cases}
    \end{align*}
\end{example}
\begin{example}[4]
    On étudie :$\displaystyle\int_{-\infty}^{+\infty}\frac{1}{1+x^2}dx$. On coupe en $w = 0$ : 
    \begin{align*}
        \int_{-\infty}^{+\infty}\frac{1}{1+x^2}dx &=\lim_{u\to -\infty}\Big[\arctan(x)\Big]_{x = u}^0 + \lim_{u\to +\infty}\Big[\arctan(x)\Big]^{x = u}_0\\
        &= \pi
    \end{align*}
\end{example}
\begin{remark}
    Si $\int_{-\infty}^{+\infty}f(x)dx$ converge, i.e. si les deux limites existent, alors cette inégrae vaut : \[\underbrace{\lim_{u\to + \infty}\int_{-u}^{u} f(x)dx}_{\textstyle\text{veleur de cauchy}}\]
\end{remark}
\begin{example}[5]
    \begin{align*}
        \int_{-\infty}^{+\infty} xdx &= \int_{-\infty}^{0} xdx + \int_{0}^{+\infty}xdx\\
        &= \lim_{u\to -\infty}\frac{-u^2}{2} + \lim_{v\to +\infty}\frac{v^2}{2}
    \end{align*}
    L'intégrale diverge ! En revenche, sa valeur principale de Cauchy est $\lim_{u\to +\infty}\int_{-u}^{u}xdx = 0$. On voit alors que la valeur principale de Cauchy $\neq \int_{-\infty}^{+\infty}$.
\end{example}
\begin{property}[Comparaison d'intégrales]
    Soient $f, g : [a, b[\to\R$ sontinues telles que $0\leq f(x) \leq g(x)$ pour tout $x \in [a, b[$. Alors : 
    \begin{enumerate}
        \item $\label{prop:comp_int_1}\int_{a}^{b^-} f(x)dx$ converge $\impliedby \int_{a}^{b ^-}g(x)dx$ converge.
            \begin{proof}
                Théorème du gendarme seul!
            \end{proof}
    \end{enumerate}
\end{property}
\begin{example}[du \ref{prop:comp_int_1}]
    $\displaystyle\int_{0}^{1^-}\frac{1}{\sqrt{1-t^2}}dt$ converge : 
    \begin{align*}
        0 \leq \frac{1}{\sqrt{1-t^2}}dt \leq \frac{1}{\sqrt{1-t}}
        \intertext{Et on a :}
        \int_{0}^{1^-} \frac{1}{\sqrt{1-t}}dt \overset{x = 1-t}{=} -\int_{1}^{0^+}\frac{1}{\sqrt{x}}dx\\
        = \int_{0^+}^{1}\frac{1}{\sqrt{x}}dx 
        \intertext{Qui converge}
    \end{align*}
\end{example}
\begin{property}[Comparaison intégrales / séries]
    Soit $f: [n_0, +\infty[ \to \R$ une fonction continue, positive, et décroissante. Alors :
    \[\sum_{n = n_0}^{\infty}f(n) \text{ converge } \iff \int_{n_ 0}^{+\infty}f(x)dx \text{ converge}\]
    \textcolor{red}{\faExclamationTriangle}\textcolor{red}{Valeurs pas égales ! }
\end{property}
\begin{example}
    La somme $\displaystyle\sum_{n = 1}^{\infty}\frac{1}{n^p}$ converge $\displaystyle\iff \int_{1}^{+\infty} \frac{1}{n^p}dx$ converge si et seulement si $p>1$.
\end{example}
\begin{example}
    \begin{align*}
        \sum_{n =2}^{\infty} \frac{1}{n(\underbrace{\log(n)}_{u})^p} \; \text{ converge } &\iff \int_{2}^{+\infty}\frac{1}{x\left(\log(x)\right)^p}dx \; \text{ converge}
        \intertext{Voyons la veleur de l'intégrale indéfinie :}
        \int\frac{1}{x\left(\log(x)\right)^p}dx &= \int\frac{1}{u^p}du = \begin{cases}
            \frac{u^{1-p}}{1-p} + c & p \neq 1\\
            \log(u) & p = 1
        \end{cases} \\
        &\overset{du = \frac{1}{x}dx}{=} \begin{cases}
            \frac{log(x)^{1-p}}{1-p} & p\neq 1\\
            \log(\log(x)) & p = 1
        \end{cases}\\
        &\implies I = \int_{2}^{+\infty} \frac{1}{x\left(\log(x)\right)^p}dx \overset{p = 1}{=} \lim_{u \to +\infty}\Big[\log(\log(x))\Big]_{x = 2}^u = +\infty\\
        &\implies I \overset{p\neq 1}{=} \lim_{u\to +\infty} \Big[\frac{\log(x)^{1-p}}{1-p}\Big]_{x = 2}^u =c\frac{-\lim_{u\to + \infty}\log(x)^{1-p}}{1-p} - \frac{\log(2)^{1-p}}{1-p}
        \intertext{$I$ converge si et seulement si $p>1$, ainsi }
        &\sum_{n =2}^{\infty}\frac{1}{n\left(\log(n)\right)^p} \text{ converge } \iff p > 1
        \intertext{En particulier : }
        &\sum_{n = 2}^{\infty}\frac{1}{n\log(n)} \text{ diverge}\\
        &\sum_{n = 2}^{\infty}\frac{1}{n\log(n)^2} \text{ converge}
    \end{align*}
\end{example}

% % !TEX root = ../Analyse1.tex
\chapter{Dériviées}
\section{Définitions et exemples}
\begin{definition}[Dériviée]
    Soit $f: D\to\R$ définie au voisinage de $x_0$ ou en $x_0$. Alors $f$ est $\underline{\text{dérivable}}$  ou $\underline{\text{différentiable}}$ en $x_0$ si
    la limite
    \[f'(x_0)=\lim_{h\to 0}\frac{f(x_0+h) -f(x_0)}{h}\]
    existe ($\in \R$).\\
    \textbf{Notation :} \[f'(x_0) = \frac{df}{dx}(x_0) = \partial_xf(x_0) =\mathcal{D}_x f(x_0) = \dot{f}(x_0)\]
    On dit :\begin{itemize}
        \item $f'(x_0)$ est la dérivée de $f$ en $x_0$\\
        \item $f:D\to\R$ est $\underline{\text{dérivable}}$ si elle est dérivable en tout $x_0 \in D$.
    \end{itemize}
\end{definition}
\begin{remark}
    Le nombre $f'(x_0)$ est la pentes de la tengente à la courbe $y=f(x)$ au point$(x_0, f(x_0))$.
\end{remark}
\begin{example}
    \begin{align*}
        f'(x_0)&=\lim_{h\to 0}\frac{f(x_0+h) -f(x_0)}{h}\\
        &=\lim_{x\to x_0} \frac{f(x)-f(x_0)}{x-x_0}
    \end{align*}
\end{example}
\begin{definition}[La fonction dérivée]
    La $\underline{\text{fonction dérivée }}$ d'une fonction $f:D\to\R$ est la fonction \(\begin{aligned}[t]
        f':D(f(x))&\longrightarrow\R\\
        x&\longmapsto f'(x)
    \end{aligned}\) où $D(f') = \left\{x\in D | f\text{ est dérivable en } x\right\}$
\end{definition}
\begin{example}[1]
    $f(x) = x^2$. On a \begin{align*}
        f'(x_0) &=\lim_{h\to 0}\frac{f(x_0+h) -f(x_0)}{h}\\
        &=\lim_{h\to 0}\frac{(x_0+h)^2 -x_0^2}{h}\\
        &= \lim_{h\to 0}(2x_0 +h) = 2\cdot x_0.
    \end{align*}
    Ainsi, $f(x)=x^2$ est dérivable pour tout $x_0\in\R$. Sa dérivée est $f'(x) = 2x$.
\end{example}
\begin{example}[2]
    $f(x)=\sin(x)$, $x_0\in \R$. On a\begin{align*}
        f'(x_0) =& \lim_{h\to 0}\frac{\sin(x_0+h)-\sin(x_0)}{h}\\ 
        =&\lim_{h\to 0}\frac{\sin(x_0)\cos(h)+\cos(x_0)\sin(h)-\sin(x_0)}{h}\\ 
        =& \sin(x_0)\cdot\lim_{h\to 0} \frac{\cos(h)-1}{h} + \cos(x_0)\cdot\lim_{h\to 0}\frac{\sin(h)}{h}\\ 
        \implies& \underbrace{-h}_{\to 0} = \frac{1-h^2-1}{h}\geq\underbrace{\frac{\cos(h)-1}{h}}_{\to 0}\geq \frac{0}{h} = \underbrace{0}_{\to 0}\\
        \implies& \sin(x_0)\cdot 0+\cos(x_0)\cdot1 = \cos(x_0)
    \end{align*}
    $\sin$ est dérivable sur $\R$ et $\sin'(x) = \cos(x)$. De manière analogue : $\cos'(x) = -\sin(x)$²
\end{example}
\begin{property}
    Soit $f:D\to\R$
    \begin{enumerate}
        \item Si $f$ est dérivable en $x_0$, alors $f$ est aussi continue en $x_0$
        \item $f$ est dérivable en $x_0$ si et seulement si :\[f(x) = \underbrace{f(x_0)+f'(x_0)\cdot(x-x_0)}_{\text{équation de la tengente}}+ \underbrace{\textcolor{red}{(x-x_0)\cdot\varepsilon(x)}}_{\text{reste}}\]
            où $\varepsilon(x)$ est une fonction tel que $\lim_{x\to x_0} \varepsilon(x)=0$. Le \textcolor{red}{reste} tend plus vite vers $0$ que $x-x_0$
    \end{enumerate}
\end{property}

% etc.
%\printbibliography

\end{document}
