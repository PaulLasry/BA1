% !TEX root = Analyse1.tex
% --- Encodage et Langue ---
\usepackage[utf8]{inputenc}
\usepackage[T1]{fontenc}
\usepackage[french]{babel}
\usepackage{geometry}
\geometry{left=2cm, right=2cm, top=2.5cm, bottom=2.5cm}

% --- Packages Mathématiques ---
\usepackage{amsmath}
\usepackage{amssymb}
\usepackage{amsfonts}

% --- Packages Graphiques et Listes ---
\usepackage{graphicx}
\usepackage{enumitem}
\setlist{itemsep=5pt, parsep=5pt}
\setlist[itemize,1]{label=\textbullet}

% --- Packages de Style (NOUVELLE MÉTHODE) ---
\usepackage[svgnames]{xcolor}
\usepackage{hyperref}
\usepackage{amsthm}      % Cœur du système de théorèmes
\usepackage{thmtools}    % Pour styliser
\usepackage[framemethod=default]{mdframed} % Pour les encadrés

% --- 1. Définition de vos couleurs ---
\definecolor{customRed}{HTML}{D9534F}
\definecolor{customOrange}{HTML}{F0AD4E}
\definecolor{customBlue}{HTML}{428BCA}
\definecolor{customViolet}{HTML}{9370DB}

\mdfdefinestyle{defstyle}{
    linecolor=customRed,
    linewidth=1.5pt,
    backgroundcolor=customRed!5, % Fond très léger
    skipabove=10pt,
    skipbelow=10pt,
    roundcorner=5pt,
    nobreak=true
}

\mdfdefinestyle{thmstyle}{
    linecolor=customViolet,
    linewidth=1.5pt,
    backgroundcolor=customViolet!5,
    skipabove=10pt,
    skipbelow=10pt,
    roundcorner=5pt,
    nobreak=true
}

\mdfdefinestyle{propstyle}{
    linecolor=customBlue,
    linewidth=1.5pt,
    backgroundcolor=customBlue!5,
    skipabove=10pt,
    skipbelow=10pt,
    roundcorner=5pt,
    nobreak=true
}

% --- 2. Configuration des Environnements (MÉTHODE STABLE) ---

% --- A. Définition des styles ---

% Style pour les textes (Rem, Cor, Prop)
\declaretheoremstyle[
    headfont=\bfseries\color{customViolet},
    bodyfont=\normalfont,
    headpunct={\ :}, % Ajoute " :" après le titre
    notefont=\normalfont\itshape, % <-- VOTRE DEMANDE : Titre optionnel en italique
    spaceabove=6pt,
    spacebelow=6pt
]{textstyle_violet}

\declaretheoremstyle[
    headfont=\bfseries\color{customBlue},
    bodyfont=\normalfont,
    headpunct={\ :},
    notefont=\normalfont\itshape, % <-- VOTRE DEMANDE : Titre optionnel en italique
    spaceabove=6pt,
    spacebelow=6pt
]{textstyle_blue}

% Style pour les boîtes (Def, Thm, Prop)
\declaretheoremstyle[
    headfont=\bfseries\color{customRed},
    bodyfont=\normalfont,
    headpunct={\ .}, % Point pour les defs/thms
    notefont=\normalfont\itshape, % <-- VOTRE DEMANDE : Titre optionnel en italique
    spaceabove=6pt,
    spacebelow=6pt
]{textstyle_red_boxed}

\declaretheoremstyle[
    headfont=\bfseries\color{customViolet},
    bodyfont=\normalfont,
    headpunct={\ .}, % Point pour les defs/thms
    notefont=\normalfont\itshape, % <-- VOTRE DEMANDE : Titre optionnel en italique
    spaceabove=6pt,
    spacebelow=6pt
]{textstyle_violet_boxed}

\declaretheoremstyle[
    headfont=\bfseries\color{customBlue},
    bodyfont=\normalfont,
    headpunct={\ .}, % Point pour les defs/thms
    notefont=\normalfont\itshape, % <-- VOTRE DEMANDE : Titre optionnel en italique
    spaceabove=6pt,
    spacebelow=6pt
]{textstyle_blue_boxed}

% Style pour les Exemples (gère "Ex 1:")
\declaretheoremstyle[
    headfont=\bfseries\color{customOrange},
    bodyfont=\normalfont,
    headpunct={\ :}, % Ajouté pour la cohérence
    spaceabove=6pt,
    spacebelow=6pt,
    notebraces = {}{} % Supprime les parenthèses autour du "1"
]{textstyle_orange}


% --- B. Création des environnements ---

% Compteur principal partagé, réinitialisé à chaque \subsection
\newtheorem{maincounter}{}[section] 

% ENCADRÉS (avec mdframed)
% On applique les styles "boxed" que nous venons de définir
\declaretheorem[name=Définition, numberlike=maincounter, style=textstyle_red_boxed]{definition}
\declaretheorem[name=Théorème, numberlike=maincounter, style=textstyle_violet_boxed]{theorem}
\declaretheorem[name=Propriété, numberlike=maincounter, style=textstyle_blue_boxed]{property}

% Application des styles de boîtes mdframed
\surroundwithmdframed[style=defstyle]{definition}
\surroundwithmdframed[style=thmstyle]{theorem}
\surroundwithmdframed[style=propstyle]{property}

% TEXTES SIMPLES (avec thmtools)
% Les styles sont déjà appliqués ici
\declaretheorem[name=Proposition, numberlike=maincounter, style=textstyle_violet]{proposition}
\declaretheorem[name=Corollaire, numberlike=maincounter, style=textstyle_violet]{corollary}
\declaretheorem[name=Remarque, numberlike=maincounter, style=textstyle_blue]{remark}

% EXEMPLE (non numéroté, gestion "Ex 1:")
\declaretheorem[
    name=Ex, 
    style=textstyle_orange, 
    numbered=no,
]{example}

%Commande perso
\newcommand{\N}{\mathbb{N}}
\newcommand{\Z}{\mathbb{Z}}
\newcommand{\Q}{\mathbb{Q}}
\newcommand{\R}{\mathbb{R}}
\newcommand{\C}{\mathbb{C}}

\usepackage[
    backend=biber,     % Utilise biber (plus moderne que bibtex)
    style=apa,         % LE STYLE APA 7 !
    sorting=nyt        % Trie par nom, puis année (standard APA)
]{biblatex}

\addbibresource{sources.bib}
% --- FIN DU FICHIER PREAMBULE ---